% Ctrl + alt + b to build & preview (Linux)
\documentclass[11pt,a4paper]{article}

\usepackage[margin=1in, paperwidth=8.3in, paperheight=11.7in]{geometry}
\usepackage{amsfonts}
\usepackage{amsmath}
\usepackage{enumerate}
\usepackage{enumitem}
\usepackage{fancyhdr}

\begin{document}

\pagestyle{fancy}
\allowdisplaybreaks

\renewcommand{\headrulewidth}{0pt}
\newcommand{\vect}[1]{\boldsymbol{#1}}
\newcommand{\subtitle}[2]{\textbf{#1}\textit{#2} \\}
\newcommand{\dotprod}[0]{\boldsymbol{\cdot}}
\setlength\parindent{0pt}

% Cover page title
\title{Calculus 1 - Notes}
\author{Dom Hutchinson}
\date{\today}
\maketitle

% Header
\fancyhead[L]{Dom Hutchinson}
\fancyhead[C]{Calculus 1 - Notes}
\fancyhead[R]{\today}

\tableofcontents

% Start of content
\newpage

\section{Before Calculus}

\subsection{Fundamental Theorem of Calculus}
\subtitle{Definition 1.01 - }{Fundamental Theorem of Calculus}
The Fundamental Theorem of Calculus states $$\frac{d}{dx}\int_{a}^{x} f(t) dt = f(x) $$ \\

\subtitle{Definition 1.02 - }{Common Sets of Numbers}
Natural Numbers, set of positive integers - $\mathbb{N} := \{1, 2, 3, ...\}$. \\
Whole Numbers, set of all integers - $\mathbb{Z} := \{..., -2, -1, 0, 1, 2, ...\}$. \\
Rational Numbers, set of fractions - $\mathbb{Q} := \left\{\frac{p}{q} : p \in \mathbb{Z}, q \in \mathbb{N} \right\}$. \\
Real Numbers, set of all rational \& irrational numbers - $\mathbb{R}$. \\

\subsection{Intervals}
\subtitle{Definition 1.03 - }{Intervals}
Sets of real numbers that fulfil in given ranges. \\
\underline{Notation}
\begin{eqnarray*}
  [a,b] := \{x \in \mathbb{R} : a \leq x \leq b\} \\
  (a,b] := \{x \in \mathbb{R} : a < x \leq b\} \\
  \left[a,b\right) := \{x \in \mathbb{R} : a \leq x < b\} \\
  (a,b) := \{x \in \mathbb{R} : a < x < b \}
\end{eqnarray*}

\underline{Example} \\
In what interval does x lie such that: $$|3x+4|<|2x-1|$$ \\
\textit{Solution}
\begin{align*}
  \mathrm{Case\ 1:}\ x \geq \frac{1}{2} & \\
  &=> 1 - 2x < 3x + 4 < 2x - 1 \\
  &=> 1 - 2x < 3x + 4 \\
  &=> x > \frac{-3}{5} \\ \\
  \mathrm{And,} &=> 3x + 4 < 2x - 1 \\
  &=> x < -5 \\
  \text{There are no real solutions in this range.} \\
  \mathrm{Case\ 2:}\ x < \frac{1}{2} & \\
  &=> 2x - 1 < 3x + 4 < 1 -2x \\
  &=> 2x - 1 < 3x + 4 \\
  &=> -5 < x \\
  \mathrm{And,} &=> 3x + 4 < 1 - 2x \\
  &=> 5x < -3 \\
  &=> x < \frac{-3}{5} \\
  %
  \\ &=>-5 < x < \frac{-3}{5},\ \underline{ x \in \left(-5,\frac{-3}{5}\right) }
\end{align*}

\subtitle{Definition 1.04 - }{Functions}
Functions map values between fields of numbers. The signature of a function is defined by $$f : A \to B$$
Where $f$ is the name of the function, $A$ is the domain and $B$ is the co-domain. \\
The \textit{Domain} of a function is the set of numbers it can take as an input. \\
The \textit{Co-Domain} is the set of numbers that the domain is mapped to. \\

\underline{N.B.} - A function is valid iff it maps each value in the domain to a single value in the co-domain. \\

\subtitle{Definition 1.05 - }{Maximal Domain}
The \textit{Maximal Domain} of a function is the largest set of values which can serve as the domain of a function. \\

\subtitle{Remark 1.06 - }{Types of Function}
Let $f:A \to B$ \\
\textit{Polynomials} $$f(x) = a_0 + a_1x + ... +a_nx^n$$
\textit{Rational} $$f(x) = \frac{p(x)}{q(x)},\quad q(x) \not = 0\ \forall\ x \in A$$
\textit{Trigonometric} $$sin(x),\ cos(x),\ tan(x)\ \mathrm{etc.}$$
%
\section{Limits}
\subsection{Limits}
\subtitle{Definition 2.01 - }{Limits}
A limit is the value a function tends to, for a given x. \\
\textit{i.e.} The value f(x) has at it gets very close to x. \\
\\\textit{Formally} We say $L$ is the limit of $f(x)$ as $x$ tends to $x_0$ if $$\forall\ \varepsilon > 0,\ \exists\ \delta > 0\ \mathrm{st\ if}\ x \in A\ \mathrm{and}\ |x - x_0| < \delta => |f(x) - L| < \varepsilon$$
\textit{Notation} $$\lim_{x \to x_0} f(x) = L$$

\newpage
\subtitle{Definition 2.02 - }{Directional Limits}
Sometimes the value of a limit depends on which direction you approach it from.\\
$\lim_{x \to x_{0}+}$ is used when approaching from values greater than $x_0$. \\
$\lim_{x \to x_{0}-}$ is used when approaching from values less than $x_0$. \\

\subtitle{Theorem 2.03 - }{Operations with limits}
Let $\lim_{x \to x_0} f(x) = L_f$ and $\lim_{x \to x_0} g(x) = L_g$
Then
\begin{alignat*}{2}
  &\lim_{x \to x_0} \left[f(x) + g(x)\right] &&= L_f + L_g \\
  &\lim_{x \to x_0} f(x).g(x) &&= L_f.L_g \\
  &\lim_{x \to x_0} \frac{f(x)}{g(x)} &&= \frac{L_f}{L_g} \quad L_g \not = 0
\end{alignat*}

\subsection{Exponential Function}
\subtitle{Definition 2.04 - }{Exponential Function}
$$e := \lim_{x \to \infty} \left(1+\frac{1}{n}\right)^n \simeq 2.7182818...$$\\

\subtitle{Theorem 2.05 - }{Binomial Expansion}
A techique for expanding binomial expressions
\begin{alignat*}{2}
\left(1+\frac{x}{n}\right)^n &= \sum_{i=0}^{n} \binom{i}{n} . 1^{(n-i)} . \left(\frac{x}{n}\right)^i \\
&= 1 + x + \frac{n-1}{2n}.x^2 + ... + \frac{x^n}{n^n}
\end{alignat*}

\section{The Derivative}

\subtitle{Definition 3.01 - }{Differentiable Equations}
Let $f : A \to B$ and $x_0 \in A$. \\
$f$ is differentiable at $x_0$ if $\exists\ L \in B$ such that $$L = \lim_{h \to 0} \frac{f(x_0 + h)-f(x_0)}{h}$$
If this limit exists $\forall\ x \in A$ then we can define the derivative of $f(x)$ $$f'(x) := \lim_{h \to 0} \frac{f(x + h) -f(x)}{h}$$

\subtitle{Definition 3.02 - }{Notation for Differentiation}
There are two ways to denote the derivative of an equation $$f'(x) \iff \frac{df}{dx}, f''(x) \iff \frac{d^2f}{dx^2}, ... , f^{(n)}(x) \iff \frac{d^nf}{dx^n}$$
\underline{N.B.} - Using $\displaystyle{\frac{df}{dx}}$ is more informative, especially for equations with multiple variables.

\subsection{Techniques for finding derivative}
%
\subtitle{Theorem 3.03 - }{Sum Rule}
Let $f, g$ be differentiable with respect to x.
$$(f+g)' = f' + g'$$

\subtitle{Theorem 3.04 - }{Product Rule}
Let $f, g$ be differentiable with respect to x.
$$(fg)' = f'g + fg'$$

\subtitle{Theorem 3.05 - }{Quotient Rule}
Let $f, g$ be differentiable with respect to x.
$$\left(\frac{f}{g}\right)' = \frac{f'g - fg'}{g^2}$$

\subtitle{Definition 3.06 - }{Composite Functions}
Let $f : B \to C$ and $g : A \to B$ Then $$(f \circ g)(x) = f(g(x))$$

\subtitle{Theorem 3.07 - }{Chain Rule}
Let $f, g$ be differentiable with respect to x.
$$\frac{d}{dx} f(g(x)) = f'(g(x)).g'(x)$$

\subsection{Implicit Differentiation}
%
\subtitle{Definition 3.08 - }{Implicit Differentiation}
Sometimes it is hard to isolate variables in multi-variable equations, in these cases differentiate both sides with respect to the same variable. \\
Remembering $$\frac{d}{dx}(x) = 1\ \mathrm{and}\ \frac{d}{dx}(y) = \frac{dy}{dx} = y'$$
\textit{Example} \\
Find $y$ if $x^3 + y^3 = 6xy$
\begin{alignat*}{2}
  &\frac{d}{dx}\left(x^3 + y^3\right) &&= \frac{d}{dx}\left(6xy\right) \\
  => &3x^2 + 3y^2.y' &&= 6y + 6x.y' \\
  => &y'(3y^2 - 6x) &&= 6y - 3x^2 \\
  => &y' &&= \frac{2y - x^2}{y^2 - 2x}
\end{alignat*}

\newpage
\subsection{Applications of The Derivative}
%
\subtitle{Thoerem 3.09 - }{Netwon's Method}
Let $f$ be differentiable. Using \textit{Newton's Method} we can approximate a solution to $f(x) = 0$.
\begin{enumerate}[label=\roman*)]
  \item Take an inital guess, $x_0$;
  \item Find the value of $x$ where the tangent to $x_0$ on $f(x)$ intercepts the x-axis;
  \item Use this value as the next guess;
  \item Repeat until the value of $x$ reduces little.
\end{enumerate}
The equation for the tangent is $$y = f(x_0) + (x - x_0)f'(x_0)$$ so a simplified equation for the process can be deduced $$x_{n+1} = x_n - \frac{f(x_n)}{f'(x_n)}$$

\subtitle{Theorem 3.10 - }{Angle between Intersecting Curves}
Let $y = f_1(x)$ and $y = f_2(x)$ be two curves which intersect at $(x_0, y_0)$. \\
Then $y_0 = f_1(x_0) = f_2(x_0)$ \\
Let $m_1, m_2$ be the gradient of the tangents to $f_1$ \& $f_2$ at $x_0$. \\
Then $\theta_i := tan^{-1}(m_i)$ for $i = 1, 2$. \\
Let $\phi = |\theta_1 - \theta_2|$, then $$\phi = \left|\frac{m_1 - m_2}{1 + m_1m_2}\right|$$

\subtitle{Theorem 3.11 - }{L'Hospital's Rule}
For two equations equations, $f,g$ with limit of $-\infty, 0$ or $\infty$ as $x$ tends to $a$, it is hard to solve $$\lim_{x \to a}\frac{f(x)}{g(x)}$$
Provided the limit exists, L'Hospital's Rule states that $$\lim_{x \to a}\frac{f(x)}{g(x)} \iff \lim_{x \to a}\frac{f'(x)}{g'(x)}$$

\subsection{Sketching Curves}
%
\subtitle{Remark 3.11 - }{Sketching Curves}
Evaluating the derivative of a curve can make it easier to sketch:
\begin{enumerate}[label=\roman*)]
  \item When $f'(x) > 0$ the curve is heading upwards;
  \item When $f'(x) < 0$ the curve is heading downwards;
  \item When $f'(x) = 0$ the curve is flat;
  \item When $f'(x) = \infty, -\infty$ there are assymptotes.
\end{enumerate}

\subtitle{Definition 3.12 - }{Even Functions}
If $f(x) = f(-x)$ then the function is symmetrical and said to be \textit{even}. \\
\textit{Examples} - $x^2, cos(x), |x|$ \\

\subtitle{Definition 3.13 - }{Odd Functions}
If $f(x) = -f(-x)$ then the function is said to be \textit{odd}. \\
\textit{Examples} - $x, sin(x), x.cos(x)$\\

\subtitle{Remark 3.14}{}
Some functions are neither \textit{odd} nor \textit{even}. \\
\textit{Example} - $x + x^2$

\section{Inegration}
%
\subsection{The Primitive}

\subtitle{Definition 4.01 - }{The Primitive}
A function, $F : A \to \mathbb{R}$, is a primative for the function $f : A \to \mathbb{R}$ if $F$ is differentiable and $$\displaystyle{\frac{d}{dx} F = f}$$
\underline{N.B.} - Primitives are also called \textit{Indefinite Integral} or \textit{Anti-Derivative}. \\

\subtitle{Remark 4.02 - }{Area Under a Curve}
Let $f : [a,b] \to \mathbb{R}$ be a continuous function. Then the area between the curve and the x-axis is found by integration. $$A := \int_{a}^{b} f(x) dx$$

\subtitle{Definition 4.03 - }{Convergent Improper Integrals}
Let $b > a$ and define a function, $f : \left[a, \infty\right) \to \mathbb{R}$, which is continuous in $[a, b]$ Then
$$\int_{0}^{\infty} f(x) dx := \lim_{b \to \infty} \int_{a}^{b} f(x) dx$$
If this limit exists then the improper integral is \textit{convergent}, otherwise it is \textit{divergent}. \\

\subtitle{Definition 4.04 - }{Definite Integral}
Let $F$ be the primative for the function $f$. Then $$\int_{b}^{a} f(x) dx = F(a) - F(b)$$
\underline{Notation} - $F(x)\big|_{a}^{b} = F(b) - F(a)$\\

\subtitle{Remark 4.05 - }{Summing Definiate Inegrals}
For all $a < c < b$ $$\int_{a}^{b} f(x) dx = \int_{c}^{a} f(x) dx + \int_{b}^{c} f(x) dx$$ $$\displaystyle{\int_{b}^{a} f(x) d := -\int_{a}^{b} f(x) dx}$$

\subtitle{Theorem 4.06 - }{Taylor Series}
Functions can be expanded into polynomial form with degree $n$, $T_n$, and remainder $R_n$ such that $f(x) = T_n(x) + R_n(x)$.
$$T_n(x) = f(a) + (x-a)f'(a) + ... + \frac{1}{n}.(x-a)^n.f^n(a)$$
$$R_n(x) = \frac{1}{n} \int_{a}^{x} (x-t)^n.f^{(n+1)}(t) dt$$

\end{document}
