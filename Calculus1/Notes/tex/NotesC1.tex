% Ctrl + alt + b to build & preview (Linux)
\documentclass[11pt,a4paper]{article}

\usepackage[margin=1in, paperwidth=8.3in, paperheight=11.7in]{geometry}
\usepackage{amsfonts}
\usepackage{amsmath}
\usepackage{fancyhdr}
\usepackage{enumitem}

\begin{document}

\pagestyle{fancy}
\allowdisplaybreaks

\renewcommand{\headrulewidth}{0pt}
\newcommand{\vect}[1]{\boldsymbol{#1}}
\newcommand{\subtitle}[2]{\textbf{#1}\textit{#2} \\}
\newcommand{\dotprod}[0]{\boldsymbol{\cdot}}
\setlength\parindent{0pt}

% Cover page title
\title{Calculus 1 - Notes}
\author{Dom Hutchinson}
\date{\today}
\maketitle

% Header
\fancyhead[L]{Dom Hutchinson}
\fancyhead[C]{Calculus 1 - Notes}
\fancyhead[R]{\today}

\tableofcontents

% Start of content
\newpage

\section{Before Calculus}

\subsection{Fundamental Theorem of Calculus}
\subtitle{Definition 1.01 - }{Fundamental Theorem of Calculus}
The Fundamental Theorem of Calculus states $$\frac{d}{dx}\int_{a}^{x} f(t) dt = f(x) $$ \\

\subtitle{Definition 1.02 - }{Common Sets of Numbers}
Natural Numbers, set of positive integers - $\mathbb{N} := \{1, 2, 3, ...\}$. \\
Whole Numbers, set of all integers - $\mathbb{Z} := \{..., -2, -1, 0, 1, 2, ...\}$. \\
Rational Numbers, set of fractions - $\mathbb{Q} := \left\{\frac{p}{q} : p \in \mathbb{Z}, q \in \mathbb{N} \right\}$. \\
Real Numbers, set of all rational \& irrational numbers - $\mathbb{R}$. \\

\subsection{Intervals}
\subtitle{Definition 1.03 - }{Intervals}
Sets of real numbers that fulfil in given ranges. \\
\underline{Notation}
\begin{eqnarray*}
  [a,b] := \{x \in \mathbb{R} : a \leq x \leq b\} \\
  (a,b] := \{x \in \mathbb{R} : a < x \leq b\} \\
  \left[a,b\right) := \{x \in \mathbb{R} : a \leq x < b\} \\
  (a,b) := \{x \in \mathbb{R} : a < x < b \}
\end{eqnarray*}

\underline{Example} \\
In what interval does x lie such that: $$|3x+4|<|2x-1|$$ \\
\textit{Solution}
\begin{align*}
  \mathrm{Case\ 1:}\ x \geq \frac{1}{2} & \\
  &=> 1 - 2x < 3x + 4 < 2x - 1 \\
  &=> 1 - 2x < 3x + 4 \\
  &=> x > \frac{-3}{5} \\ \\
  \mathrm{And,} &=> 3x + 4 < 2x - 1 \\
  &=> x < -5 \\
  \text{There are no real solutions in this range.} \\
  \mathrm{Case\ 2:}\ x < \frac{1}{2} & \\
  &=> 2x - 1 < 3x + 4 < 1 -2x \\
  &=> 2x - 1 < 3x + 4 \\
  &=> -5 < x \\
  \mathrm{And,} &=> 3x + 4 < 1 - 2x \\
  &=> 5x < -3 \\
  &=> x < \frac{-3}{5} \\
  %
  \\ &=>-5 < x < \frac{-3}{5},\ \underline{ x \in \left(-5,\frac{-3}{5}\right) }
\end{align*}

\subtitle{Definition 1.04 - }{Functions}
Functions map values between fields of numbers. The signature of a function is defined by $$f : A \to B$$
Where $f$ is the name of the function, $A$ is the domain and $B$ is the co-domain. \\
The \textit{Domain} of a function is the set of numbers it can take as an input. \\
The \textit{Co-Domain} is the set of numbers that the domain is mapped to. \\

\underline{N.B.} - A function is valid iff it maps each value in the domain to a single value in the co-domain. \\

\subtitle{Definition 1.05 - }{Maximal Domain}
The \textit{Maximal Domain} of a function is the largest set of values which can serve as the domain of a function. \\

\subtitle{Remark 1.06 - }{Types of Function}
Let $f:A \to B$ \\
\textit{Polynomials} $$f(x) = a_0 + a_1x + ... +a_nx^n$$
\textit{Rational} $$f(x) = \frac{p(x)}{q(x)},\quad q(x) \not = 0\ \forall\ x \in A$$
\textit{Trigonometric} $$sin(x),\ cos(x),\ tan(x)\ \mathrm{etc.}$$
%
\section{Limits}
\subsection{Limits}
\subtitle{Definition 2.01 - }{Limits}
A limit is the value a function tends to, for a given x. \\
\textit{i.e.} The value f(x) has at it gets very close to x. \\
\\\textit{Formally} We say $L$ is the limit of $f(x)$ as $x$ tends to $x_0$ if $$\forall\ \varepsilon > 0,\ \exists\ \delta > 0\ \mathrm{st\ if}\ x \in A\ \mathrm{and}\ |x - x_0| < \delta => |f(x) - L| < \varepsilon$$
\textit{Notation} $$\lim_{x \to x_0} f(x) = L$$

\newpages
\subtitle{Remark 2.02 - }{Common Limits}
\begin{alignat*}{2}
  &\lim_{x \to x_0} \frac{1-x}{1+x} &= 1 \\
  &\lim_{x \to \infty} \frac{1}{x} &= 0 \\
  &\lim_{x \to 0} \frac{sin(x)}{x} &= 1
\end{alignat*} \\

\subtitle{Theorem 2.03 - }{Operations with limits}
Let $\lim_{x \to x_0} f(x) = L_f$ and $\lim_{x \to x_0} g(x) = L_g$
Then
\begin{alignat*}{2}
  &\lim_{x \to x_0} \left[f(x) + g(x)\right] &&= L_f + L_g \\
  &\lim_{x \to x_0} f(x).g(x) &&= L_f.L_g \\
  &\lim_{x \to x_0} \frac{f(x)}{g(x)} &&= \frac{L_f}{L_g} \quad L_g \not = 0
\end{alignat*}

\subsection{Exponential Function}
\subtitle{Definition 2.04 - }{Exponential Function}
$$e := \lim_{x \to \infty} \left(1+\frac{1}{n}\right)^n \simeq 2.7182818...$$\\

\subtitle{Theorem 2.05 - }{Binomial Expansion}
A techique for expanding binomial expressions
\begin{alignat*}{2}
\left(1+\frac{x}{n}\right)^n &= \sum_{i=0}^{n} \binom{i}{n} . 1^{(n-i)} . \left(\frac{x}{n}\right)^i \\
&= 1 + x + \frac{n-1}{2n}.x^2 + ... + \frac{x^n}{n^n}
\end{alignat*}

\end{document}
