% Ctrl + alt + b to build & preview (Linux)
\documentclass[11pt,a4paper]{article}

\usepackage[margin=1in, paperwidth=8.3in, paperheight=11.7in]{geometry}
\usepackage{amsfonts}
\usepackage{amsmath}
\usepackage{fancyhdr}
\usepackage{enumitem}

\begin{document}

\pagestyle{fancy}

\renewcommand{\headrulewidth}{0pt}
\newcommand{\vect}[1]{\boldsymbol{#1}}
\newcommand{\dotprod}[0]{\boldsymbol{\cdot}}

\setlength\parindent{0pt}

% Cover page title
\title{Linear Algebra \& Geometry - Notes}
\author{Dom Hutchinson}
\date{\today}
\maketitle

% Header
\fancyhead[L]{Dom Hutchinson}
\fancyhead[C]{Linear Algebra \& Geometry - Notes}
\fancyhead[R]{\today}

\tableofcontents

% Start of content
\newpage

\section{Euclidean Plane, Vectors, Cartesian Co-Ordinates \& Complex Numbers}

\subsection{Vectors}

\textbf{Definition 1.01 - }\textit{Vectors} \\
Ordered sets of real numbers. \\
Denoted by $\vect{v} = (v_1, v_2, v_3,...) = \begin{pmatrix} x \\ y \end{pmatrix}$ \\

\textbf{Definition 1.02 - }\textit{Euclidean Plane} \\
The set of two dimensional vectors, with real componenets, is called the Euclidean Plane. \\
Denoted by $\mathbb{R}^2$ \\

\textbf{Definition 1.03 - }\textit{Vector Addition} \\
Let $\vect{v}, \vect{w} \in \mathbb{R}^2$ such that $\vect{v} = (v_1,v_2)$ and $\vect{w} = (w_1,w_2)$. \\
Then $\vect{v} + \vect{w} = (v_1 + w_1, v_2 + w_2)$. \\

\textbf{Definition 1.03 - }\textit{Scalar Multiplcation of Vectors} \\
Let $\vect{v} \in \mathbb{R}^2$ and $\lambda \in \mathbb{R}$ such that $\vect{v} = (v_1,v_2)$. \\
Then $\lambda\vect{v} = (\lambda v_1, \lambda v_2)$. \\

\textbf{Definition 1.04 - }\textit{Norm of vectors} \\
The norm of a vector is its length from the origin. \\
Denoted by $||\vect{v}|| = \sqrt{v_{1}^{2} + v_{2}^{2}}$ for $\vect{v} \in \mathbb{R}^2$. \\

\textbf{Theorem 1.05} \\
Let $\vect{v}, \vect{w} \in \mathbb{R}^2$ and $\lambda \in \mathbb{R} $ such that $\vect{v} = (v_1,v_2)$ and $\vect{w} = (w_1,w_2)$. \\
%
\textbf{1)} $$||\vect{v}|| = 0\ \mathrm{iff}\ \vect{v} = \vect{0}$$ \\
%
\textbf{2)}
\begin{align*}
  ||\lambda\vect{v}|| &= \sqrt{\lambda^2v_{1}^{2} + \lambda^2v_{2}^{2}} \\
  &= |\lambda|.||\vect{v}||
\end{align*}
%
\textbf{3)} $$||\vect{v} + \vect{w}|| \leq ||\vect{v}|| + ||\vect{w}||$$ \\

\textbf{Definition 1.06 - }\textit{Unit Vector} \\
A vector can be described by its length \& direction. \\
Let $\vect{v} \in \mathbb{R}^2\backslash\{\vect{0}\}$. \\
Then $\vect{v} = ||\vect{v}||\vect{u}$ where $\vect{u}$ is the unit vector, $\vect{u} = \begin{pmatrix} cos\theta \\ sin\theta \end{pmatrix}$ \\
Thus $\forall\ \vect{v} \in \mathbb{R}^2\ \vect{v} = \begin{pmatrix} \lambda cos\theta \\ \lambda sin\theta \end{pmatrix}$ for some $\lambda \in \mathbb{R}$. \\

\textbf{Definition 1.07 - }\textit{Dot Product} \\
Let $\vect{v} \in \mathbb{R}^2$ and $\lambda \in \mathbb{R}$ such that $\vect{v} = (v_1,v_2)$. \\
Then $\vect{v} \dotprod \vect{w} = v_1.w_1 + v_2.w_2$. \\

\textbf{Remark 1.08 - }\textit{Positivity of Dot Product} \\
Let $\vect{v} \in \mathbb{R}^2$. \\
Then $\vect{v} \dotprod \vect{v} = ||\vect{v}||^2 = v_{1}^{2} + v_{2}^{2} \geq 0$. \\

\textbf{Remark 1.09 - }\textit{Angle between vectors in Euclidean Plane} \\
Let $\vect{v}, \vect{w} \in \mathbb{R}^2$. \\
Set $\theta$ to be the angle between $\vect{v}$ \& $\vect{w}$. \\
Then $$cos\theta = \frac{\vect{v} \dotprod \vect{w}}{||\vect{v}||\ ||\vect{w}||}$$.

\textbf{Theorem 1.10 - }\textit{Cauchy-Schwarz Inequality} \\
Let $\vect{v}, \vect{w} \in \mathbb{R}^2$. \\
Then $$|\vect{v} \dotprod \vect{w}| \leq ||\vect{v}||\ ||\vect{w}||$$
%
\textit{Proof}
\begin{align*}
  \frac{v_1w_1}{||\vect{v}||\ ||\vect{w}||} + \frac{v_2w_2}{||\vect{v}||\ ||\vect{w}||} &\leq \frac{1}{2}\left(\frac{v_{1}^{2}}{||\vect{v}||^2} + \frac{w_{1}^{2}}{||\vect{w}||^2}\right) + \frac{1}{2}\left(\frac{v_{2}^{2}}{||\vect{v}||^2} + \frac{w_{2}^{2}}{||\vect{w}||^2}\right) \\
  &\leq \frac{1}{2}\left(\frac{v_{1}^{2} + v_{2}^{2}}{||\vect{v}||^2} + \frac{w_{1}^{2} + w_{2}^{2}}{||\vect{w}||^2}\right) \\
  &\leq \frac{1}{2}(1 + 1) \\
  &\leq 1 \\
  => |v_1w_1 + v_2w_2| &\leq ||\vect{v}||\ ||\vect{w}|| \\
  |\vect{v} \dotprod \vect{w}| &\leq ||\vect{v}||\ ||\vect{w}|| \\
\end{align*}

\end{document}
