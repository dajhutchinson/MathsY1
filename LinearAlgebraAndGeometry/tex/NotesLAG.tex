% Ctrl + alt + b to build & preview (Linux)
\documentclass[11pt,a4paper]{article}

\usepackage[margin=1in, paperwidth=8.3in, paperheight=11.7in]{geometry}
\usepackage{amsfonts}
\usepackage{amsmath}
\usepackage{fancyhdr}
\usepackage{enumitem}

\begin{document}

\pagestyle{fancy}
\setlength\parindent{0pt}
\allowdisplaybreaks

\renewcommand{\headrulewidth}{0pt}
\newcommand{\vect}[1]{\boldsymbol{#1}}
\newcommand{\subtitle}[2]{\textbf{#1}\textit{#2} \\}
\newcommand{\dotprod}[0]{\boldsymbol{\cdot}}

% Cover page title
\title{Linear Algebra \& Geometry - Notes}
\author{Dom Hutchinson}
\date{\today}
\maketitle

% Header
\fancyhead[L]{Dom Hutchinson}
\fancyhead[C]{Linear Algebra \& Geometry - Notes}
\fancyhead[R]{\today}

\tableofcontents

% Start of content
\newpage

\section{Euclidean Plane, Vectors, Cartesian Co-Ordinates \& Complex Numbers}

\subsection{Vectors}

\subtitle{Definition 1.01 - }{Vectors}
Ordered sets of real numbers. \\
Denoted by $\vect{v} = (v_1, v_2, v_3,...) = \begin{pmatrix} x \\ y \end{pmatrix}$ \\

\subtitle{Definition 1.02 - }{Euclidean Plane}
The set of two dimensional vectors, with real componenets, is called the Euclidean Plane. \\
Denoted by $\mathbb{R}^2$ \\

\subtitle{Definition 1.03 - }{Vector Addition}
Let $\vect{v}, \vect{w} \in \mathbb{R}^2$ such that $\vect{v} = (v_1,v_2)$ and $\vect{w} = (w_1,w_2)$. \\
Then $\vect{v} + \vect{w} = (v_1 + w_1, v_2 + w_2)$. \\

\subtitle{Definition 1.03 - }{Scalar Multiplcation of Vectors}
Let $\vect{v} \in \mathbb{R}^2$ and $\lambda \in \mathbb{R}$ such that $\vect{v} = (v_1,v_2)$. \\
Then $\lambda\vect{v} = (\lambda v_1, \lambda v_2)$. \\

\subtitle{Definition 1.04 - }{Norm of vectors}
The norm of a vector is its length from the origin. \\
Denoted by $||\vect{v}|| = \sqrt{v_{1}^{2} + v_{2}^{2}}$ for $\vect{v} \in \mathbb{R}^2$. \\

\subtitle{Theorem 1.05}{}
Let $\vect{v}, \vect{w} \in \mathbb{R}^2$ and $\lambda \in \mathbb{R} $ such that $\vect{v} = (v_1,v_2)$ and $\vect{w} = (w_1,w_2)$. \\
Then
\begin{align*}
  ||\vect{v}|| &= 0\ \mathrm{iff}\ \vect{v} = \vect{0}\\
  ||\lambda\vect{v}|| &= \sqrt{\lambda^2v_{1}^{2} + \lambda^2v_{2}^{2}} \\
  &= |\lambda|.||\vect{v}|| \\
  ||\vect{v} + \vect{w}|| &\leq ||\vect{v}|| + ||\vect{w}||
\end{align*} \\

\subtitle{Definition 1.06 - }{Unit Vector}
A vector can be described by its length \& direction. \\
Let $\vect{v} \in \mathbb{R}^2\backslash\{\vect{0}\}$. \\
Then $\vect{v} = ||\vect{v}||\vect{u}$ where $\vect{u}$ is the unit vector, $\vect{u} = \begin{pmatrix} cos\theta \\ sin\theta \end{pmatrix}$ \\
Thus $\forall\ \vect{v} \in \mathbb{R}^2\ \vect{v} = \begin{pmatrix} \lambda cos\theta \\ \lambda sin\theta \end{pmatrix}$ for some $\lambda \in \mathbb{R}$. \\

\subtitle{Definition 1.07 - }{Dot Product}
Let $\vect{v} \in \mathbb{R}^2$ and $\lambda \in \mathbb{R}$ such that $\vect{v} = (v_1,v_2)$. \\
Then $\vect{v} \dotprod \vect{w} = v_1.w_1 + v_2.w_2$. \\

\subtitle{Remark 1.08 - }{Positivity of Dot Product}
Let $\vect{v} \in \mathbb{R}^2$. \\
Then $\vect{v} \dotprod \vect{v} = ||\vect{v}||^2 = v_{1}^{2} + v_{2}^{2} \geq 0$. \\

\subtitle{Remark 1.09 - }{Angle between vectors in Euclidean Plane}
Let $\vect{v}, \vect{w} \in \mathbb{R}^2$. \\
Set $\theta$ to be the angle between $\vect{v}$ \& $\vect{w}$. \\
Then $$cos\theta = \frac{\vect{v} \dotprod \vect{w}}{||\vect{v}||\ ||\vect{w}||}$$.

\subtitle{Theorem 1.10 - }{Cauchy-Schwarz Inequality}
Let $\vect{v}, \vect{w} \in \mathbb{R}^2$. \\
Then $$|\vect{v} \dotprod \vect{w}| \leq ||\vect{v}||\ ||\vect{w}||$$
%
\textit{Proof}
\begin{align*}
  \frac{v_1w_1}{||\vect{v}||\ ||\vect{w}||} + \frac{v_2w_2}{||\vect{v}||\ ||\vect{w}||} &\leq \frac{1}{2}\left(\frac{v_{1}^{2}}{||\vect{v}||^2} + \frac{w_{1}^{2}}{||\vect{w}||^2}\right) + \frac{1}{2}\left(\frac{v_{2}^{2}}{||\vect{v}||^2} + \frac{w_{2}^{2}}{||\vect{w}||^2}\right) \\
  &\leq \frac{1}{2}\left(\frac{v_{1}^{2} + v_{2}^{2}}{||\vect{v}||^2} + \frac{w_{1}^{2} + w_{2}^{2}}{||\vect{w}||^2}\right) \\
  &\leq \frac{1}{2}(1 + 1) \\
  &\leq 1 \\
  => |v_1w_1 + v_2w_2| &\leq ||\vect{v}||\ ||\vect{w}|| \\
  |\vect{v} \dotprod \vect{w}| &\leq ||\vect{v}||\ ||\vect{w}|| \\
\end{align*}

\subsection{Complex Numbers}

\subtitle{Definition 1.11 - }{i}
\begin{alignat*}{1}
  i^2 &= -1 \\
  i &= \sqrt{-1}
\end{alignat*}

\subtitle{Definition 1.12 - }{Complex Number Set}
The set of complex numbers contains all numbers with an imaginary part. $$\mathbb{C} := \left\{x + iy; x,y \in \mathbb{R}\right\}$$
Complex numbers are often denoted by $$z = x + iy$$ and we say $x$ is the real part of $z$ and $y$ the imaginary part.

\subtitle{Definition 1.13 - }{Complex Conjugate}
Let $z \in \mathbb{C}$ st $z = x + iy$. \\
Then $$\bar{z} := x - iy$$
%
\newpage
%
\subtitle{Theorem 1.14 - }{Operations on Complex Numbers}
Let $z_1,z_2 \in \mathbb{C}$ st $z_1 = x_1 + iy_1$ and $z_2 = x_2 + iy_2$. \\
Then \begin{alignat*}{2}
  &z_1 + z_2 &&:= (x_1 + x_2) + i(y_1 + y_2) \\
  &z_1.z_2 &&:= (x_1 + iy_1)(x_2 + iy_2) \\
  & && := x_1.x_2 - y_1.y_2 + i(x_1.y_2 + x_2.y_1)
\end{alignat*}
\underline{N.B.} When dividing by a complex number, multiply top and bottom by the complex conjugate. \\

\subtitle{Definition 1.15 - }{Modulus of Complex Numbers}
The modulus of a complex number is the distance of the number, from the origin, on an Argand diagram.
Let $z \in \mathbb{C}$ st $z = x + iy$. \\
Then \begin{alignat*}{2}
  |z| &:= \sqrt{x^2 + y^2} \\
  &:= \sqrt{\bar{z}z}
\end{alignat*}
\underline{N.B.} Amplitude is an alternative name for the modulus \\

\subtitle{Definition 1.16 - }{Phase of Complex Numbers}
The phase of a complex number is the angle between the positive real axis and the line subtended from the origin and the number, on an Argand digram.$$z = |z|.(cos\theta + i.sin\theta), \quad \theta = \mathrm{Phase}$$
\underline{N.B.} Phase of $\bar{z}$ = - Phase of $z$ \\

\subtitle{Theorem 1.17 - }{de Moivre's Formula}
$$z^n = (cos\theta +i.sin\theta)^n = cos(n\theta)+i.sin(n\theta)$$

\subtitle{Theorem 1.18 - }{Euler's Formula}
$$e^{i\theta} = cos\theta + i.sin\theta$$

\subtitle{Remark 1.19}{}
Using Euler's formula we can express all complex numbers in terms of $e$. Thus many properties of the exponential remain true:
\begin{alignat*}{2}
  z &= \lambda e^{i\theta}, && \quad \lambda \in \mathbb{R} , \theta \in \left[0, 2\pi\right) \\
  => z_1 + z_2 &= \lambda_1 . \lambda_2. e^{i(\theta_1 + \theta_2)} \\
  \&, \frac{z_1}{z_2} &= \frac{\lambda_1}{\lambda_2}.e^{i(\theta_1 = \theta_2)}
\end{alignat*}

\end{document}
