% Ctrl + alt + b to build & preview (Linux)
\documentclass[11pt,a4paper]{article}

\usepackage[margin=1in, paperwidth=8.3in, paperheight=11.7in]{geometry}
\usepackage{amsfonts}
\usepackage{amsmath}
\usepackage{enumerate}
\usepackage{enumitem}
\usepackage{fancyhdr}

\begin{document}

\pagestyle{fancy}
\setlength\parindent{0pt}
\allowdisplaybreaks

\renewcommand{\headrulewidth}{0pt}
\newcommand{\vect}[1]{\boldsymbol{#1}}
\newcommand{\subtitle}[2]{\textbf{#1}\textit{#2} \\}
\newcommand{\dotprod}[0]{\boldsymbol{\cdot}}
\newcommand{\real}[0]{\mathbb{R}}
\newcommand{\nat}[0]{\mathbb{N}}
\newcommand{\field}[0]{\mathbb{F}}
\newcommand{\vectorspace}[0]{\mathbb{V}}
\newcommand{\basis}[0]{\mathbb{B}}
\newcommand{\complex}[0]{\mathbb{C}}

% Cover page title
\title{Linear Algebra \& Geometry - Notes}
\author{Dom Hutchinson}
\date{\today}
\maketitle

% Header
\fancyhead[L]{Dom Hutchinson}
\fancyhead[C]{Linear Algebra \& Geometry - Notes}
\fancyhead[R]{\today}

\tableofcontents

% Start of content
\newpage

\section{Euclidean Plane, Vectors, Cartesian Co-Ordinates \& Complex Numbers}

\subsection{Vectors}

\subtitle{Definition 1.01 - }{Vectors}
Vectors are ordered sets of real numbers.\\
Denoted by $\vect{v} = (v_1, v_2, v_3,...)$.\\

\subtitle{Definition 1.02 - }{Euclidean Plane}
The set of two dimensional vectors, with real componenets, is called the Euclidean Plane. \\
Denoted by $\real^2$.\\

\subtitle{Definition 1.03 - }{Vector Addition}
Let $\vect{v}, \vect{w} \in \real^2$ such that $\vect{v} = (v_1,v_2)$ and $\vect{w} = (w_1,w_2)$. Then\\
$$\vect{v} + \vect{w} = (v_1 + w_1, v_2 + w_2)$$

\subtitle{Definition 1.03 - }{Scalar Multiplcation of Vectors}
Let $\vect{v} \in \real^2$ and $\lambda \in \real$ such that $\vect{v} = (v_1,v_2)$. Then\\
$$\lambda\vect{v} = (\lambda v_1, \lambda v_2)$$

\subtitle{Definition 1.04 - }{Norm of vectors}
The norm of a vector is its length from the origin. \\
$$||\vect{v}|| := \sqrt{v_{1}^{2} + v_{2}^{2}}, \quad \vect{v} \in \real^2$$

\subtitle{Theorem 1.05 - }{Properties of the Norm}
Let $\vect{v}, \vect{w} \in \real^2$ and $\lambda \in \real $ such that $\vect{v} = (v_1,v_2)$ and $\vect{w} = (w_1,w_2)$. Then
\begin{align*}
  ||\vect{v}|| &= 0\ \mathrm{iff}\ \vect{v} = \vect{0}\\
  ||\lambda\vect{v}|| &= \sqrt{\lambda^2v_{1}^{2} + \lambda^2v_{2}^{2}} \\
  &= |\lambda|.||\vect{v}|| \\
  ||\vect{v} + \vect{w}|| &\leq ||\vect{v}|| + ||\vect{w}||
\end{align*}

\subtitle{Definition 1.06 - }{Unit Vector}
A vector can be described by its length \& direction. \\
Let $\vect{v} \in \real^2\backslash\{\vect{0}\}$. \\
Then $\vect{v} = ||\vect{v}||.\vect{u}$ where $\vect{u}$ is the unit vector, $\vect{u} = \begin{pmatrix} cos\theta \\ sin\theta \end{pmatrix}$ \\
Thus $\forall\ \vect{v} \in \real^2, \vect{v} = \begin{pmatrix} \lambda cos\theta \\ \lambda sin\theta \end{pmatrix}$ for some $\lambda \in \real\ \&\ \vect{w} = (w_1, w_2)$. \\

\subtitle{Definition 1.07 - }{Dot Product}
Let $\vect{v}, \vect{w} \in \real^2$ and $\lambda \in \real$ where $\vect{v} = (v_1,v_2)\ \&\ \vect{w} = (w_1, w_2)$. \\
Then $\vect{v} \dotprod \vect{w} = v_1.w_1 + v_2.w_2$. \\

\subtitle{Remark 1.08 - }{Positivity of Dot Product}
Let $\vect{v} \in \real^2$. \\
Then $\vect{v} \dotprod \vect{v} = v_{1}^{2} + v_{2}^{2} = ||\vect{v}||^2 \geq 0$. \\

\subtitle{Remark 1.09 - }{Angle between vectors in Euclidean Plane}
Let $\vect{v}, \vect{w} \in \real^2$. \\
Set $\theta$ to be the angle between $\vect{v}$ \& $\vect{w}$. \\
Then
$$cos(\theta) = \frac{\vect{v} \dotprod \vect{w}}{||\vect{v}||\ ||\vect{w}||}$$.

\subtitle{Theorem 1.10 - }{Cauchy-Schwarz Inequality}
Let $\vect{v}, \vect{w} \in \real^2$. \\
Then $$|\vect{v} \dotprod \vect{w}| \leq ||\vect{v}||\ ||\vect{w}||$$

\textit{Proof}
\begin{align*}
  \frac{v_1w_1}{||\vect{v}||\ ||\vect{w}||} + \frac{v_2w_2}{||\vect{v}||\ ||\vect{w}||} &\leq \frac{1}{2}\left(\frac{v_{1}^{2}}{||\vect{v}||^2} + \frac{w_{1}^{2}}{||\vect{w}||^2}\right) + \frac{1}{2}\left(\frac{v_{2}^{2}}{||\vect{v}||^2} + \frac{w_{2}^{2}}{||\vect{w}||^2}\right) \\
  &\leq \frac{1}{2}\left(\frac{v_{1}^{2} + v_{2}^{2}}{||\vect{v}||^2} + \frac{w_{1}^{2} + w_{2}^{2}}{||\vect{w}||^2}\right) \\
  &\leq \frac{1}{2}(1 + 1) \\
  &\leq 1 \\
  => |v_1w_1 + v_2w_2| &\leq ||\vect{v}||\ ||\vect{w}|| \\
  |\vect{v} \dotprod \vect{w}| &\leq ||\vect{v}||\ ||\vect{w}|| \\
\end{align*}

\subsection{Complex Numbers}

\textbf{Definition 1.11 - }\textit{i}
\begin{alignat*}{1}
  i^2 &= -1 \\
  i &= \sqrt{-1}
\end{alignat*}

\subtitle{Definition 1.12 - }{Complex Number Set}
The set of \textit{complex numbers} contains all numbers with an imaginary part. $$\complex := \left\{x + iy; x,y \in \real\right\}$$
Complex numbers are often denoted by $$z = x + iy$$ and we say $x$ is the real part of $z$ and $y$ the imaginary part.

\subtitle{Definition 1.13 - }{Complex Conjugate}
Let $z \in \complex$ st $z = x + iy$. Then
$$\bar{z} := x - iy$$

\subtitle{Theorem 1.14 - }{Operations on Complex Numbers}
Let $z_1,z_2 \in \complex$ st $z_1 = x_1 + iy_1$ and $z_2 = x_2 + iy_2$. Then
\begin{alignat*}{2}
  &z_1 + z_2 &&:= (x_1 + x_2) + i(y_1 + y_2) \\
  &z_1.z_2 &&:= (x_1 + iy_1)(x_2 + iy_2) \\
  & && := x_1.x_2 - y_1.y_2 + i(x_1.y_2 + x_2.y_1)
\end{alignat*}
\underline{N.B.} When dividing by a complex number, multiply top and bottom by the complex conjugate. \\

\subtitle{Definition 1.15 - }{Modulus of Complex Numbers}
The \textit{modulus} of a complex number is the distance of the number, from the origin, on an Argand diagram.
Let $z \in \complex$ st $z = x + iy$. Then
$$|z| := \sqrt{x^2 + y^2} = \sqrt{\bar{z}z}$$
\underline{N.B.} \textit{Amplitude} is an alternative name for the modulus \\

\subtitle{Definition 1.16 - }{Phase of Complex Numbers}
The phase of a complex number is the angle between the positive real axis and the line subtended from the origin and the number, on an Argand digram.$$z = |z|.(cos\theta + i.sin\theta), \quad \theta = \mathrm{Phase}$$
\underline{N.B.} (Phase of $\bar{z}$) = - (Phase of $z$) \\

\subtitle{Theorem 1.17 - }{de Moivre's Formula}
$$z^n = (cos(\theta) +i.sin(\theta))^n = cos(n\theta)+i.sin(n\theta)$$

\subtitle{Theorem 1.18 - }{Euler's Formula}
$$e^{i\theta} = cos(\theta) + i.sin(\theta)$$

\subtitle{Remark 1.19}{}
Using Euler's formula we can express all complex numbers in terms of $e$. Thus many properties of the exponential remain true:
\begin{alignat*}{2}
  z &= \lambda e^{i\theta}, && \quad \lambda \in \real , \theta \in \left[0, 2\pi\right) \\
  => z_1 + z_2 &= \lambda_1 . \lambda_2. e^{i(\theta_1 + \theta_2)} \\
  \&, \frac{z_1}{z_2} &= \frac{\lambda_1}{\lambda_2}.e^{i(\theta_1 = \theta_2)}
\end{alignat*}

\section{Euclidean Space, $\real^n$}

\subtitle{Definition 2.01 - }{Euclidean Space}
Let $n \in \nat$ then $\forall\ \vect{x} = (x_1, x_2, ... , x_n)$ with $x_1, x_2, ... , x_n \in \real$ then $\vect{x} \in \real^n$. \\

\subtitle{Theorem 2.02 - }{Operations in Euclidean Space}
Let $\vect{x}, \vect{y} \in \real^n$ and $\lambda \in \real$.
Then
$$\vect{x} + \vect{y} = (x_1 + y_1, ... , x_n + y_n)$$
And
$$\vect{x} + \lambda.\vect{y} = (x_1 + \lambda.y_1, ... , x_n + \lambda.y_n)$$

\subtitle{Definition 2.03 - }{Cartesian Product}
Let $A, B \in \real^n$ be non-empty sets. \\
Then $$A \times B := \{(a,b); a \in A, b \in B\}$$

\subsection{Dot Product}
%
\subtitle{Definition 2.04 - }{Dot Product}
Let $\vect{v}, \vect{w} \in \real^n$. Then \begin{alignat*}{1}
\vect{v} \dotprod \vect{w} &:= v_1.w_1 + ... + v_n.w_n \\
&:= \sum_{j=1}^n v_j.w_j
\end{alignat*}

\subtitle{Theorem 2.05 - }{Properties of the Dot Product}
Let $\vect{u}, \vect{v}, \vect{w} \in \real^n$.\\
Linearity:
$$(\vect{u} + \lambda\vect{v}) \dotprod \vect{w} = \vect{u} \dotprod \vect{w} + \lambda(\vect{v} \dotprod \vect{w})$$
Symmetry:
$$\vect{v} \dotprod \vect{w} = \vect{w} \dotprod \vect{v}$$
Positivity:
$$\vect{v} \dotprod \vect{v} = v_1^2 + v_2^2 + ... +v_n^2 \geq 0$$

\subtitle{Definition 2.06 - }{Orthogonality}
Let $\vect{v}, \vect{w} \in \real^n$. \\
It is said that $\vect{v}, \vect{w}$ are \textit{orthogonal} to each other if $\vect{v} \dotprod \vect{w} = 0$ \\
\underline{N.B.} Orthogonal vectors are perpendicular to each other. \\

\subtitle{Definition 2.07 - }{The Norm}
Let $\vect{x} \in \real^n$. Then
$$||\vect{x}|| = \sqrt{\vect{x} \dotprod \vect{x}} = \sqrt{\sum_{i=1}^n x_i^2}$$

\subtitle{Theorem 2.08 - }{Properties of the Norm}
Let $\vect{x}, \vect{y} \in \real^n$ and $\lambda \in \real$. Then
\begin{alignat*}{2}
  &||\vect{x}|| &&\geq 0 \\
  &||\vect{x}|| &&= 0 \mathrm{\ iff\ } \vect{x} = 0 \\
  &||\lambda\vect{x}|| &&= |\lambda| ||\vect{x}|| \\
  &||\vect{x} + \vect{y}|| &&\leq ||\vect{x}|| + ||\vect{y}||
\end{alignat*}

\subtitle{Theorem 2.09 - }{Dot Product and Norm}
Let $\vect{x}, \vect{y} \in \real^n$.
$$|\vect{x} \dotprod \vect{y}| \leq ||\vect{x}|| ||\vect{y}||$$
\underline{N.B.} $|\vect{x} \dotprod \vect{y}| = ||\vect{x}||.||\vect{y}||$ iff $\vect{x}$ \& $\vect{y}$ are orthogonal.\\

\subtitle{Theorem 2.10 - }{Angle between Vectors}
Let $\vect{x}, \vect{y} \in \real^n$. Then
$$cos(\theta) = \frac{\vect{x} \dotprod \vect{y}}{||\vect{x}||.||\vect{y}||}$$

\subsection{Linear Subspaces}

\subtitle{Definition 2.11 - }{Linear Subspace}
Let $V \subset \real^n$.\\
$V$ is a \textit{Linear Subspace} if
\begin{enumerate}[label=\roman*)]
  \item $V \not = \emptyset$;
  \item $\forall\ \vect{v}, \vect{w} \in V$ then $\vect{v} + \vect{w} \in V$; and
  \item $\forall\ \lambda \in \real, \vect{v} \in V$ then $\lambda\vect{v} \in V$.\\
\end{enumerate}

\subtitle{Definition 2.12 - }{Span}
Let $\vect{x_1}, ... , \vect{x_k} \in \real^n, k \in \nat$. Then
$$\mathrm{span}\{\vect{x_1}, ... , \vect{x_k}\} := \{\lambda_1\vect{x_1} + ... + \lambda_k\vect{x_k}; \lambda_i \in \real, 0 \leq i \geq k\}$$

\subtitle{Theorem 2.13 - }{Spans are Subspaces}
Let $\vect{x_1}, ... , \vect{x_k} \in \real^n;\ k \in \nat$. Then span$\{\vect{x_1}, ... , \vect{x_k}\}$ is a linear subspace of $\real^n$. \\

\subtitle{Theorem 2.14}{}
$$W_{\vect{a}} := \{\vect{x} \in \real^n; \vect{x} \dotprod \vect{a} = 0\}\mathrm{\ is\ a\ subspace.}$$

\subtitle{Definition 2.15 - }{Orthogonal Complement}
Let $V \subset \real^n$. Then
$$V^{\perp} := \{\vect{x} \in \real^n; \vect{x} \dotprod \vect{y}\ \forall\ \vect{y} \in V\}$$
\underline{N.B.} $V^{\perp} \subset \real^n$.\\

\subtitle{Theorem 2.16 - }{Relationship of Subspaces}
Let $V, W \subset \real^n$. Then
$$V \cap W \mathrm{\ is\ a\ subspace\ and}$$
$$V+W := \{\vect{v} + \vect{w}; \vect{v} \in V, \vect{w} \in W\}\mathrm{\ is\ a\ subspace.}$$

\subtitle{Definition 2.17 - }{Direct Sum}
Let $V_1, V_2, W$ be subspaces of $\real$. Then $W$ is said to be a \textit{direct sum} if
\begin{enumerate}[label=\roman*)]
  \item $W = V_1 + V_2$; and,
  \item $V_1 \cap V_2 = \emptyset$.
\end{enumerate}

\section{Linear Equations \& Matrices}

\subsection{Linear Equations}

\subtitle{Definition 3.01 - }{Multi-Variable Linear Equations}
Linear equations produce a straight line and can have multiple variables. \\
\textit{Examples} $x = 3, y = x + 3, z + 5x - 2y$ \\

\subtitle{Defintion 3.02 - }{Systems of Linear Equations}
Let $\vect{a}, \vect{x} \in \real^n$ \& $b \in \real$ such that $\vect{a} \dotprod \vect{x} = b$. \\
$\vect{a} \dotprod \vect{x} = b$ is a linear equation in $S$ with $S(\vect{a}, b) = \{\vect{x}; \vect{a} \dotprod \vect{x} = b\}$ as the set of solutions. \\
\underline{N.B.} If $b = 0$ then $S(\vect{a}, 0)$ is a subspace.

\subsection{Matrices}

\subtitle{Definition 3.03 - }{Matrix}
Let $m, n \in \nat$, then a $m \times n$ grid of numbers form an \textit{'m by n'} matrix.
Each element of the matrix can be reference by $a_{ij}$ with $i = 1, ... , m$ and $j = 1, ... , n$.
$$ A = \begin{pmatrix}
  a_{11} & a_{12} & ... & a_{1n} \\
  a_{21} & a_{22} & ... & a_{2n} \\
  \vdots & \vdots & \ddots & \vdots \\
  a_{m1} & a_{m2} & ... & a_{mn}
\end{pmatrix}
$$
\underline{N.B.} m\ \&\ i = rows; n\ \&\ j = columns \\

\subtitle{Definition 3.04 - }{Sets of Matrices}
$M_{m,n}(\real)$ is the set of m x n matrices containing only real elements. \\
$M_{m,n}(\mathbb{Z})$ is the set of m x n matrices containing only integer elements. \\
$M_{n}(\real)$ is the set square matrices, size n, containing only real elements. \\

\subtitle{Definition 3.05 - }{Transpose Vectors}
Let $\vect{x} = \begin{pmatrix} x_1 \\ x_2 \\ \vdots \\ x_n\end{pmatrix}$ then $\vect{x}^t = \begin{pmatrix} x_1 & x_2 & ... & x_n\end{pmatrix}$\\\\

\subtitle{Definition 3.06 - }{Vector-Matrix Multiplication}
Let $A \in \real_{m,n}$ and $\vect{x} \in \real^n$ then
$$A\vect{x} := \begin{pmatrix}
  a_{11} & a_{12} & ... & a_{1n} \\
  a_{21} & a_{22} & ... & a_{2n} \\
  \vdots & \vdots & \ddots & \vdots \\
  a_{m1} & a_{m2} & ... & a_{mn}
\end{pmatrix} \begin{pmatrix}
  x_1 \\ x_2 \\ \vdots \\ x_n
\end{pmatrix} = \begin{pmatrix}
  a_{11}x_1 + a_{12}x_2 + ... + a_{1n}{x_n} \\
  a_{21}x_1 + a_{22}x_2 + ... + a_{2n}{x_n} \\
  \vdots \\
  a_{m1}x_1 + a_{m2}x_2 + ... + a_{mn}{x_n} \\
\end{pmatrix} = \begin{pmatrix}
  \vect{a}_{1}^{t} \dotprod \vect{x} \\
  \vect{a}_{2}^{t} \dotprod \vect{x} \\
  \vdots \\
  \vect{a}_{m}^{t} \dotprod \vect{x} \\
\end{pmatrix} \in \real^m
$$ This can be simplified to $$\vect{y} = A\vect{x} \mathrm{\ with\ } y_i = \sum_{j=1}^n a_{ij} x_j$$

\textbf{Theorem 3.07 - }\textit{Operations on Matrices with Vectors}
\begin{enumerate}[label=\roman*)]
  \item $A(\vect{x} + \vect{y}) = A\vect{x} + A\vect{y},\quad \forall\ \vect{x}, \vect{y} \in \real^n$.
  \item $A(\lambda\vect{x}) = \lambda(A\vect{x}),\quad \forall\ \vect{x} \in \real^n, \lambda \in \real$.\\
\end{enumerate}

\subtitle{Theorem 3.08 - }{Composition of Matrices}
Let $A = (a_{ij}) \in M_{m,n}(\real)$ and $B = (b_{ij}) \in M_{l,m}(\real)$.\\
Then there exists a $C = (c_{ij}) \in M_{l,n}(\real)$ such that
$$C\vect{x} = B(A\vect{x}), \quad \forall\ \vect{x} \in \real^n$$
\underline{N.B.} $c_{ij} = \sum_{k=1}^m b_{ik} a_{kj}$ \\

\subtitle{Theorem 3.09 - }{Operations with Matrices}
Let $A, B \in M_{m,n}$ and $C \in M_{l,m}$
\begin{enumerate}[label=\roman*)]
  \item $C(A + B) = CA + CB$;
  \item $(A + B)C = AC + BC$; and,
  \item Let $D \in M_{m,n}, E \in M_{n,l}$ \& $F \in M_{l, k}$ then $$E(FG) = (EF)G$$
\end{enumerate}
\underline{N.B.} $AB \not = BA$ \\

\subtitle{Definition 3.10 - }{Types of Matrix}
Upper Triangle - $\begin{pmatrix}
1 & 2 & 3 \\
0 & 4 & 5 \\
0 & 0 & 6
\end{pmatrix},\quad a_{ij} = 0 \mathrm{\ if\ } i > j$. \\
Lower Triangle - $\begin{pmatrix}
1 & 0 & 0 \\
2 & 3 & 0 \\
4 & 5 & 6
\end{pmatrix},\quad a_{ij} = 0 \mathrm{\ if\ } i < j$. \\
Symmetric Matrix - $\begin{pmatrix}
1 & 2 & 3 \\
2 & 4 & 0 \\
3 & 0 & 1
\end{pmatrix},\quad a_{ij} = a_{ji}.$ \\
Anti-Symmetric - $\begin{pmatrix}
1 & -2 & -3 \\
2 & 0 & -4 \\
3 & 4 & -1
\end{pmatrix},\quad a_{ij} = -a_{ji}$. \\\\

\subtitle{Definition 3.11 - }{Transposed Matrices}
Let $A = (a_{ij}) \in M_{m,n}(\real)$ then the transponse of $A$, $A^t$, is an element of $M_{n,m}(\real)$.
$$A^t := (a_{ji})$$

\subtitle{Theorem 3.12 - }{Transpose Matrix Multiplication}
Let $A \in M_{m,n}(\real), \vect{x} \in \real^n\ \&\ \vect{y} \in \real^m$. Then
$$\vect{y} \dotprod A \vect{x} = \left(A_t \vect{y}\right) \dotprod \vect{x}$$

\subtitle{Theorem 3.10 - }{Transposing Multiplied Matrices}
$$(AB)^t = B^t A^t$$

\subsection{Structure of Set of Solutions}
%
\subtitle{Definition 3.13 - }{Set of Solutions}
Let $A \in M_{m,n}(\real)$ and $\vect{b} \in \real^m$. Then
$$S(A, \vect{b}) :=\left\{\vect{x} \in \real^n; A\vect{x} = b\right\}$$

\subtitle{Definition 3.14 - }{Homogenous Solutions}
The system of $S(A, \vect{0})$ is said to be \textit{homogenous}.\\
All other systems are \textit{inhomogenous}.
\underline{N.B.} - $S(A, \vect{0})$ is a linear subspace. \\

\subtitle{Theorem 3.15 - }{Using Homogenous Solutions}
Let $A \in M_{m,n}(\real)$ and $\vect{b} \in \real^n$. Let $\vect{x}_0 \in \real^n$ such that $A\vect{x}_0 = \vect{b}$, then
$$S(A, \vect{b}) = {\vect{x}_0} + S(A, \vect{0})$$

\subtitle{Remark 3.16 - }{Systems of Linear Equations as Matrices}
The system of linear equations $3x + z = 0, y - z = 1\ \&\ 3x + y = 1$ can be represented by a matrix and a vector.
$$A = \begin{pmatrix}
3 & 0 & 1 \\
0 & 1 & -1 \\
3 & 1 & 0
\end{pmatrix},\ \vect{b} = \begin{pmatrix}
0 \\ 1 \\ 1
\end{pmatrix}$$

\subsection{Solving Systems of Linear Equations}
Systems of linear equations can be displayed as matrices which can be reduced and solved by a technique called \textit{Gaussian Elimination}. \\ \\
%
\subtitle{Theorem 3.17 - }{Operations on Linear Equations}
There are certain operations that can be performed on a system of linear equations without changing the result:
\begin{enumerate}[label=\roman*)]
  \item Multiply an equaion by a non-zero constant;
  \item Add a multiple of any equation to another equation; and,
  \item Swap any two equations.\\
\end{enumerate}

\subtitle{Definition 3.18 - }{Augmented Matrices}
Let $A\vect{x} = \vect{b}$ be a system of linear equations.\\
The associated \textit{Augmented Matrix} is $$(A\ \vect{b}) \in M_{m, n+1}(\real)$$

\subtitle{Theorem 3.19 - }{Elementary Row Operations}
From \textit{Theorem 3.17} we can deduce ceratin operations that can be performed on an \textit{Augmented Matrix} which do not alter the solutions:
\begin{enumerate}[label=\roman*)]
  \item Multiply a row by a non-zero constant, $row\ i \to \lambda(row\ i)$;
  \item Add a multiple of any row to another row, $row\ i \to row\ i + \lambda(row\ j)$; and,
  \item Swap two rows, $row\ i \leftrightarrow row\ j $.\\
\end{enumerate}

\subtitle{Definition 3.20 - }{Row Echelon Form}
A matrix is in \textit{Row Echelon Form} if:
\begin{enumerate}[label=\roman*)]
  \item The left-most non-zero value in each row is $1$; And,
  \item The leading $1$ in each row is one place to the right of the leading $1$ in the row below.
\end{enumerate}
\textit{Example} $$\begin{pmatrix}
  1 & a & b \\
  0 & 1 & c \\
  0 & 0 & 1
\end{pmatrix}$$ \\

\subtitle{Definition 3.20 - }{Reduced Row Echelon Form}
A matrix is in \textit{Reduced Row Echelon Form} if:
\begin{enumerate}[label=\roman*)]
  \item The matrix is in \textit{row echelon form}; and,
  \item All values in a row, except the leading $1$, are $0$.
\end{enumerate}
\textit{Example} $$\begin{pmatrix}
  1 & 0 & 0 \\
  0 & 1 & 0 \\
  0 & 0 & 1
\end{pmatrix}$$\\

\subtitle{Theorem 3.21 - }{Gaussian Elimination}
\textit{Gaussian Elimination} is a technique used to solve systems of linear equations.
\textit{Example} \\
Solve $x + y + 2z = 9, 2x + 4y - 3z = 1, 3x + 6y - 5z = 0$.
\begin{alignat*}{2}
  \mathrm{Augmented\ Matrix\ -\ } &\begin{pmatrix}
    1 & 1 & 2 & 9 \\
    2 & 4 & -3 & 1 \\
    3 & 6 & -5 & 0
  \end{pmatrix} && \\
  \mathrm{By\ EROS\ -\ } &\begin{pmatrix}
    1 & 1 & 2 & 9 \\
    2 & 4 & -3 & 1 \\
    3 & 6 & -5 & 0
  \end{pmatrix} &&= \begin{pmatrix}
    1 & 1 & 2 & 9 \\
    0 & 2 & -7 & -17 \\
    0 & 3 & -11 & 27
  \end{pmatrix} \\ &  &&= \begin{pmatrix}
    1 & 1 & 2 & 9 \\
    0 & 2 & -7 & -17 \\
    0 & 1 & -4 & -10
  \end{pmatrix} \\ & &&= \begin{pmatrix}
    1 & 1 &2 & 9 \\
    0 & 1 & -4 & -10 \\
    0 & 2 & -7 & -17
  \end{pmatrix} \\ & &&= \begin{pmatrix}
    1 & 1 & 2 & 9 \\
    0 & 1 & -4 & -10 \\
    0 & 0 & 1 & 3
  \end{pmatrix} \\ & &&= \begin{pmatrix}
    1 & 0 & 6 & 19 \\
    0 & 1 & -4 & -10 \\
    0 & 0 & 1 & 3
  \end{pmatrix}\\ & &&= \begin{pmatrix}
    1 & 0 & 0 & 1 \\
    0 & 1 & 0 & 2 \\
    0 & 0 & 1 & 3
  \end{pmatrix}\\
  => & \underline{x = 1, y= 2, z = 3}
\end{alignat*}

\subsection{Elementary Matrices \& Inverting Matrices}
%
\subtitle{Definition 3.22 - }{Invertible Matrices}
A matrix, $A \in M_{m,n}(\real)$, is said to be \textit{invertible} if there exists $A^{-1} \in M_{n,m}(\real)$ such that
$$AA^{-1} = I$$
\underline{N.B.} - If a matrix is not invertible then it is \textit{singular}.\\

\subtitle{Definition 3.23 - }{Elementary Matrices}
A matrix, $E \in M_{m,n}(\real)$, is said to be an \textit{Elementary Matrix} if it can be obtained by performing Elementary Row Operations on a square identity matrix.\\
\textit{Example}
$\begin{pmatrix}
  0 & \lambda \\
  \mu & 0
\end{pmatrix}$ \\

\subtitle{Remark 3.24}{}
All elementary matrices are invertible.

\subtitle{Remark 3.25}{}
Let $A$ be a matrix, and $B$ be a matrix which can be obtained from $A$ by elementary row operations. Then there exists an elementary matrix, $E$, such that $$B = EA$$.

\subtitle{Theorem 3.26 - }{Finding $A^{-1}$}
Let $A = \begin{pmatrix}
  a_{11} & a_{12} & a_{13} \\
  a_{21} & a_{22} & a_{23} \\
  a_{31} & a_{32} & a_{33}
\end{pmatrix},\ B = \begin{pmatrix}
  b_{11} & b_{12} & b_{13} \\
  b_{21} & b_{22} & b_{23} \\
  b_{31} & b_{32} & b_{33}
\end{pmatrix},\ I = \begin{pmatrix}
  1 & 0 & 0 \\
  0 & 1 & 0 \\
  0 & 0 & 1
\end{pmatrix}$. \\
Then by using EROS to change $(A\ I) \to (I\ B)$, $B$ is the inverse of $A$. \\

\subtitle{Theorem 3.27 - }{Inverse of a $2 \times 2$ Matrix}
Let $A = \begin{pmatrix}
  a & b \\
  c & d
\end{pmatrix}$ then $$A^{-1} = \frac{1}{ad - bc} \begin{pmatrix}
  d & -b \\
  -c & a
\end{pmatrix} = \frac{1}{det(A)} \begin{pmatrix}
  d & -b \\
  -c & a
\end{pmatrix} $$

\section{Linear Independence, Bases \& Dimensions}
\subsection{Linear Independence \& Dependence}
%
\subtitle{Definition 4.01 - }{Linear Independence \& Dependence}
Vectors, $\vect{x}_1, ... , \vect{x}_n \in \real^k$, are said to be \textit{linearly dependent} if there exists non-zero real numbers, $\lambda_1, ... , \lambda_n$, such that
$$\lambda_1.\vect{x}_1 + ... + \lambda_n.\vect{x}_n = \vect{0}$$
\underline{N.B.} - If this is only true if $\lambda_1 = ... = \lambda_n = 0$ then the vectors are said to be \textit{linearly independent}. \\

\subtitle{Remark 4.02}{}
Vectors are \textit{linearly dependent} if at least one of them lies in the span of the rest.

\subsection{Bases \& Dimensions}
%
\subtitle{Definition 4.03 - }{Basis}
A \textit{basis} is a set of vectors, $\vect{v}_1, ... , \vect{v}_n \in V$ such that
\begin{enumerate}[label=\roman*)]
  \item $V = \mathrm{span}\{\vect{v}_1, .... , \vect{v}_n\}$; and,
  \item $\vect{v}_1, ... , \vect{v}_n$ are linearly independent.\\
\end{enumerate}

\subtitle{Definition 4.04 - }{Standard Basis}
The \textit{standard basis} for a vector space is the set fewest unit vectors which span it. \\
\textit{Example} - $\{\vect{e}_1, \vect{e}_2 , \vect{e}_3\}$ are the standard basis for $\real^3$. \\

\subtitle{Theorem 4.05 - }{Basis of a Linear Subspace}
For all elements, $\vect{v}$, of a linear subspace, $V \subset \real^n$, there exists a \underline{unique} set of numbers, $\lambda_1, ... , \lambda_n$, such that
$$\vect{v} = \lambda_1.\vect{v}_1 + ... + \lambda_n.\vect{v}_n$$

\subtitle{Theorem 4.06 - }{Linear Independence and Bases}
Let $V \subset \real^n$ be a linear subspace with basis ${\vect{v}_1, ... , \vect{v}_n}$.\\
Suppose $\vect{w}_1 , ... , \vect{w}_k \in V$ are linearly independent, then $k \leq n$.\\

\subtitle{Definition 4.07 - }{Dimension}
Let $V \subset \real^n$ be a linear subspace then the \textit{dimension} of $V$, $dim(V)$, is the fewest number vectors required to form a basis for $V$.

\subsection{Orthogonal Bases}
%
\subtitle{Definition 4.08 - }{Orthogonal}
Let $V \subset \real^n$ be a linear subspace with $\{\vect{v}_1 , ... , \vect{v}_k\}$ as its basis.\\
This basis is an \textit{orthogonal basis} if it statisfies
\begin{enumerate}[label=\roman*)]
  \item $\vect{v}_i \dotprod \vect{v}_j = 0$ if $i \not = j$; and,
  \item $\vect{v}_i \dotprod \vect{v}_i = 1,\ i = 1, ... , k$.
\end{enumerate}
\underline{N.B.} - This can be generalised to $\vect{v}_i \dotprod \vect{v}_k = \delta_{ij}$ with $\delta_{ij} := \begin{cases} 1, & i = j \\ 0, & \mathrm{otherwise}\end{cases}$\\\\

\subtitle{Theorem 4.09}{}
Let $V \subset \real^n$ be a linear subspace with an orthogonal basis $\{\vect{v}_1 , ... , \vect{v}_k\}$. \\
Then for all $\vect{u} \in V$ $$\vect{u} = (\vect{v}_1 \dotprod \vect{u}) \vect{v}_1 , ... , (\vect{v}_k \dotprod \vect{u}) \vect{v}_k$$

\section{Linear Maps}

\subtitle{Definition 5.01 - }{Linear Map}
A map, $T : \real^n \to \real^m$ is a \textit{linear map} if
\begin{enumerate}[label=\roman*)]
  \item $T(\vect{x} + \vect{y}) = T(\vect{x}) + T(\vect{y}),\quad \forall\ \vect{x}, \vect{y} \in \real^n$; and,
  \item $T(\lambda\vect{x}) = \lambda T(\vect{x}),\quad \forall\ \vect{x} \in \real^n, \lambda \in \real$.
\end{enumerate}
\underline{N.B.} - If $m = n$ then $T$ is referred to as a \textit{linear operator}. \\

\subtitle{Theorem 5.02 - }{Properties of Linear Maps}
Let $T : \real^n \to \real^m$ be a linear map. Then $T(\vect{0}) = \vect{0}$. \\

\subtitle{Definiton 5.03 - }{Linear Maps as Matrices}
Let $T : \real^n \to \real^m$ be a linear map. Then the associated Matrix is defined as
$$M_T := (t_{ij}) \in M_{m,n}(\real)$$
with the elements of $M_T$ defined by
$$t_{ij} := \vect{e}_i \dotprod T(\vect{e}_j)$$

\subtitle{Theorem 5.04 - }{Solutions to Linear Maps from Matrices}
Let $T : \real^n \to \real^m$ be a linear map and $M_T$ be the associated matrix. Then $$T(\vect{x}) = M_T\vect{x},\quad \forall\ \vect{x} \in \real^n$$

\subsection{Abstract Properties of Linear Maps}
%
\subtitle{Theorem 5.05 - }{Relationship between Linear Maps}
Let $S : \real^n \to \real^m$, $T : \real^n \to \real^m$ \& $U : \real^m \to \real^k$ be a linear maps, $\vect{x} \in \real^n$ and $\lambda \in \real$. Then
\begin{enumerate}[label=\roman*)]
  \item $(\lambda T)(\vect{x}) = \lambda T(\vect{x})$;
  \item $(S + T)(\vect{x}) = S(\vect{x}) + T(\vect{x})$; and,
  \item $(U \circ S)(\vect{x}) = U(S(\vect{x}))$.\\
\end{enumerate}

\subtitle{Definition 5.06 - }{Image \& Kernel}
Let $T : \real^n \to \real^m$ be a linear map. Then\\
The \textit{image} of $T$ is defined to be $$Im(T) := \{\vect{y} \in \real^m : \exists\ \vect{x} \in \real^n st\ T(\vect{x}) = \vect{y}\}$$
The \textit{kernel} of $T$ is defined to be $$Ker(T) := \{\vect{x} \in \real^n : T(\vect{x}) = \vect{0}\}$$

\subtitle{Theorem 5.07 - }{Image \& Kernel are Linear Subspaces}
Let $T : \real^n \to \real^m$ be a linear map then $Im(T)$ is a linear subsapce of $\real^m$ and $Ker(T)$ is a linear subspaces of $\real^n$ \\

\subtitle{Remark 5.08}{}
Let $T : \real^n \to \real^m$ be a linear map. Then
\begin{enumerate}[label=\roman*)]
  \item $T$ is surjective if $Im(T) = \real^m$; and,
  \item $T$ is injective if $Ker(T) = \{ 0\}$.
\end{enumerate}

\subsection{Matrices}

\subtitle{Definition 5.09 - }{Linear Maps as Matrices}
Let $S : \real^n \to \real^m$, $T : \real^n \to \real^m$ \& $U : \real^m \to \real^k$ be a linear maps and $\lambda \in \real$ with $M_S, M_T\ \&\ M_U$ as the corresponding matrices. Then
\begin{enumerate}[label=\roman*)]
  \item $M_{\lambda T} = \lambda M_T = (\lambda t_{ij})$;
  \item $M_{S + T} = (s_{ij} + t_{ij}) = M_S + M_T$; and,
  \item $M_{U \circ S} = (r_{ij})$ where $r_{ik} = \sum_{k=1}^m s_{ik}t_{jk}$.
\end{enumerate}

\subsection{Rank \& Nullity}

\subtitle{Defintion 5.10 - }{Rank \& Nullity}
Let $T : \real^n \to \real^m$ be a linear map. Then we define \textit{Rank} of $T$ by $$rank(T) := dim(Im(T))$$
and we define \textit{Nullity} of $T$ by $$nullity(T) := dim(Im(T))$$
\underline{N.B.} - For all linear maps, $T : \real^n \to \real^m$, $$nullity(T) + rank(T) = n$$

\subtitle{Remark 5.11}{}
Let $T : \real^n \to \real^n$ be a linear map. Then $T$ is invertible if
\begin{enumerate}[label=\roman*)]
  \item $rank(T) = n$, or
  \item $nullity(T) = 0$.
\end{enumerate}

\subtitle{Theorem 5.12 - }{Relationship of Rank \& Nullity between Linear Maps}
Let $S : \real^n \to \real^m$ \& $T : \real^k \to \real^n$ be linear maps. Then
\begin{enumerate}[label=\roman*)]
  \item $S \circ T = 0$ iff $Im(T) \subset Ker(S)$;
  \item $rank(S \circ T) \leq rank(T)$ and $rank(S \circ T) \leq rank(S)$;
  \item $nullity(S \circ T) \geq nullity(T)$ and $nullity(S \circ T) \geq nullity(S) + k - n$; and,
  \item $S$ is invertible then $rank(S \circ T) = rank(T)$ and $nullity(S \circ T) = nullity(T)$.
\end{enumerate}

\section{Determinants}

\subsection{Definition \& Basic Properties}

\subtitle{Definition 6.01 - }{Determinant Function}
A \textit{determinant function} $d_n : \real^n \times ... \times \real^n \to \real$ is a function which statisfies three conditions:
\begin{enumerate}[label=\roman*)]
  \item \textit{Multilinearity} - $d_2(\lambda\vect{a}_1 + \mu\vect{b}, \vect{a}_2) = \lambda d_2(\vect{a}_1, \vect{a}_2) + \mu(\vect{b}, \vect{a}_2)$;
  \item \textit{Antisymmetry} - $d_2(\vect{a}_1, \vect{a}_2) = -d_2(\vect{a}_2, \vect{a}_1)$; and,
  \item \textit{Normalisation} - $d_2(\vect{e}_1, \vect{e}_2) = 1$.
\end{enumerate}
\underline{N.B.} - Determinant functions only exist for \textit{square matrices}. \\

\textbf{Theorem 6.02 - }\textit{Properties of Determinant}
\begin{enumerate}[label=\roman*)]
  \item $det[ ... , \vect{a}_i + \lambda\vect{a}_i , ...] = det[ ... , \vect{a}_i , ...] + \lambda det[ ... , \vect{a}_i , ...]$;
  \item If $A$ has two identical columns then $det(A) = 0$;
  \item If $A$ has an all zero column then $det(A) = 0$; and,
  \item $det[... \vect{a}_i ... \vect{a}_j ...] = det[... (\vect{a}_i + \lambda\vect{a}_j) ... \vect{a}_j ...]$\\
\end{enumerate}

\subtitle{Theorem 6.03}{}
Let $f_n : \real^n \times ... \times \real^n \to \real$ be a function which is \textit{multilinear} \& \textit{Antisymmetric} then $$f_n(A) = C.det(A)$$
where $C$ is a constant such that $C = f_n(\vect{e}_1, ... , \vect{e}_n)$. \\

\subtitle{Theorem 6.04 - }{Determinant of a Triangle Matrix}
Let $A = (a_{ij}) \in M_{n}(\real)$ be a upper triangle matrix, so $a_{ij} = 0$ if $i > j$. Then $$det(A) = a_{11}.a_{22}.\ ...\ .a_{nn}$$
\underline{N.B.} - The same is true for lower triangle matrices. \\

\subtitle{Theorem 6.05 - }{Relationship between Determinants}
Let $A, B \in M_{n}(\real)$ then $$det(AB) = det(A).det(B)$$ but usually $$det(A + B) \not = det(A) + det(B)$$

\subtitle{Theorem 6.06 - }{Determinant \& the Inverse Matrix}
If $det(A) = 0$ then $A^{-1}$ does not exist. \\

\subtitle{Theorem 6.07 - }{Leibniz Formula}
Let $A = (a_{ij}) \in M_n(\real)$ then the \textit{Leibniz Formula} states that $$det(A) := \sum_{\sigma \in S_n} \mathrm{sign}(\sigma) \prod_{j=1}^{n}a_{\sigma(j),j}$$
Where
\begin{itemize}[label={-}]
  \item $S_n$ is the group of symmetries for a regular n-sided polygons;
  \item sign$(\sigma)$ is the sign function which returns $+1$ for even permuations and $-1$ for odd permutations. \\
  A permutation is even if a even number of permutations (swaps) are required to change the identity permutation to the given permutation, $\sigma$.\\
\end{itemize}

\subtitle{Remark 6.08 - }{Determinant of Transpose}
Let $A$ be a square matrix, then $$det(A) = det(A^t)$$

\subsection{Computing Determinant}

\subtitle{Theorem 6.09 - }{Laplace's Rule}
Let $A \in M_n$ then $$det(A) = \sum_{j=1}^n a_{ij} . (-1)^{i+j} . det(A_{ij})$$ where $A_{ij}$ is the $(n-1) \times (n-1)$ matrix formed when row $i$ and column $j$ are removed from $A$.\\
\textit{Example} Let $A = \begin{pmatrix} 1 & 2 \\ 3 & 4 \end{pmatrix}$ then $A_{11} = \begin{pmatrix} 4 \end{pmatrix}$ and $A_{12} = \begin{pmatrix} 2 \end{pmatrix}$\\\\

\subtitle{Definition 6.10 - }{Adjunct Matrices}
Let $A, B \in M_n$ be defined such that $b_{ij} = (-1)^{i+j} . det(A_ij)$ then $B$ is said to be \textit{adjunt} to $A$. \\
This means
$$AB = \begin{pmatrix} det(A) & 0 & \dots & 0 \\ 0 & det(A) & \dots & 0 \\ \vdots & \vdots & \ddots & \vdots \\ 0 & 0 & \dots & det(A) \end{pmatrix} = det(A)I$$\\

\subtitle{Remark 6.11 - }{Determinant of Triangle Matrices}
If $A$ is an upper triangle matrix ($a_{ij} = 0 \mathrm{\ if\ } i > j$) then $$det(A) = a_{11} \times \dots \times a_{nn}$$
If $A$ is a lower triangle matrix ($a_{ij} = 0 \mathrm{\ if\ } i < j$) then $$det(A) = a_{11} \times \dots \times a_{nn}$$

\subsection{Applications of Determinant}

\subtitle{Theorem 6.12 - }{Linear Equations as Matrices}
A system of $m$ linear equations, each with $n$ variables, can be written as $$A\vect{x} = \vect{b},\quad A \in M_{mn}(\real), \vect{x} \in \real^n, \vect{b} \in \real^m$$
If $det(A) \not = 0$ then we can find an $A^{-1} \in M_{n,m}$ such that $$\vect{x} = A^{-1}\vect{b}$$

\subtitle{Theorem 6.13}{}
Let $A \in M_n(\real)$ where $det(A) \not = 0$ then $$A^{-1} = \frac{1}{det(A)}adj\ A$$

\subtitle{Theorem 6.14 - }{Cramer's Rule}
Consider $A\vect{x} = \vect{b}$ then $$x_j = \frac{det(A_j)}{det(A)}$$ where $A_j$ is the matrix $A$, but the $j^{th}$ column has been replaced by $\vect{b}$. \\

\subtitle{Definition 6.15 - }{Cross Product}
Let $\vect{x}, \vect{y} \in \real^3$ be in the \underline{same plane} then we define the \textit{cross product} by
$$\vect{x} \times \vect{y} := \begin{vmatrix}
  \vect{e}_1 & \vect{e}_2 & \vect{e}_3 \\
  \vect{x}_1 & \vect{x}_2 & \vect{x}_3 \\
  \vect{y}_1 & \vect{y}_2 & \vect{y}_3
\end{vmatrix} = \begin{pmatrix}
  \vect{x}_2\vect{y}_3 - \vect{x}_3\vect{y}_2 \\
  \vect{x}_3\vect{y}_1 - \vect{x}_1\vect{y}_3 \\
  \vect{x}_1\vect{y}_2 - \vect{x}_2\vect{y}_1
\end{pmatrix}$$

\textbf{Theorem 6.16 - }\textit{Properties of Cross Product}
\begin{enumerate}[label=\roman*)]
  \item $ \vect{x} \dotprod (\vect{y} \times \vect{z}) = \vect{z} \dotprod (\vect{x} \times \vect{y}) = \vect{y} \dotprod (\vect{z} \times \vect{x}) $;
  \item $ \vect{x} \times \vect{y} = - \vect{y} \times \vect{x} $;
  \item $ \vect{x} \times \vect{x} = 0 $;
  \item $ (\vect{x} + \lambda\vect{y}) \times \vect{z} = (\vect{x} \times \vect{z}) + (\lambda\vect{y} \times \vect{z}) $; and,
  \item $ ||\vect{x} \times \vect{y}||^2 = ||\vect{x}||^2||\vect{y}||^2 - (\vect{x} \dotprod \vect{y})^2$.
\end{enumerate}

\subtitle{Theorem 6.17 - }{Cross Product and Angle between vectors}
Let $\theta$ be the angle between two vectors then
$$||\vect{x} \times  \vect{y}||^2 = ||\vect{x}||^2.||\vect{y}||^2.sin^2(\theta)$$

\subtitle{Theorem 6.18 - }{Cross Product with Matrices}
Let $A \in M_n(\real)$ where $det(A) \not = 0$ then
$$(A\vect{x}) \times (A\vect{y}) = [det(A)](A^t)^{-1}(\vect{x} \dotprod \vect{y})$$

\section{Vector Spaces}

\subsection{Groups \& Fields}

\subtitle{Definition 7.01 - }{Group}
A group, $G$, is a combination of a set and a map from $G \times G \to G$. The map must obey the following rules:
\begin{enumerate}[label=\roman*)]
  \item \textit{Associativity} - $f * (g * h) = (f * g) * h$;
  \item \textit{Identity Element} - $\exists\ e \in G\ st\ \forall\ g \in G, eg = ge = g$; and,
  \item \textit{Inverse} - $\forall\ g \in G\ \exists\ g^{-1} \in G\ st\ gg^{-1} = e = g^{-1}g$.
\end{enumerate}

\subtitle{Definition 7.02 - }{Matrix Groups}
The \textit{General Linear Group}, $GL(n, \real)$, is a group defined by $$GL(n ,\real) = \{A \in M_n(\real) : det(A) \not = 0\}$$
The \textit{identity element} is $I \in M_n$ and inverse is $A^{-1}$. \\

The \textit{Special Linear Group}, $SL(n, \real)$, is a group defined by $$SL(n ,\real) = \{A \in M_n(\real) : det(A) = 1\}$$

The \textit{Orthogonal Group}, $O(n, \real)$, is a group defined by $$O(n ,\real) = \{A \in M_n(\real) : A^t = A^{-1}\}$$

The \textit{Special Orthogonal Group}, $SO(n, \real)$, is a group defined by $$SO(n ,\real) = \{A \in O(n, \real) : det(A) = \pm 1\}$$

The \textit{Borel Matrix}, $B(n, \real)$, is the group of upper triangle matrices with non-zero values on the main diagonal.\\

The \textit{Permutations Group}, $S(n, \real)$, is a group of permutations of $\{1, 2, \dots, n\}$ defined my $n \times n$ matrix
$$\mathrm{\textit{Example}\ }\begin{pmatrix} 0 & 1 & 0 \\ 1 & 0 & 0 \\ 0 & 0 & 1 \end{pmatrix} \begin{pmatrix}1 \\ 2 \\3 \end{pmatrix} = \begin{pmatrix}2 \\ 1 \\ 3 \end{pmatrix}$$\

\subtitle{Theorem 7.03 - }{Abelian Groups}
Let $G$ be a group. If $\forall\ g, h \in G, gh = hg$ then $G$ is commutative and is called an \textit{Abelian Group}.\\
\underline{N.B.} $e = 0$ is the identity element of all Abelian groups.\\

\subtitle{Definition 7.04 - }{Direct/Cartesian Product of a Group}
Let $G, H$ be groups with the same map. Then $$G \times H := \{(g, h) : g \in G, h \in H\}$$

\subtitle{Definition 7.05 - }{Fields, $\field$}
A field, $\field$, is a set with two binary operations: addition \& multiplication.\\

\textbf{Theorem 7.06 - }\textit{Properties of Fields}
\begin{enumerate}[label=\roman*)]
  \item $\field$ is an abelian group w.r.t addition;
  \item $\field \backslash \{0\}$  is an abeelian group w.r.t multiplication;
  \item $(x + y) . z = x.z + y.z$; and,
  \item A field always contains $0$ \& $1$.
\end{enumerate}

\subsection{Vector Spaces}

\subtitle{Definition 7.07 - }{Vector Space, $\vectorspace$}
$\vectorspace$ is a (linear) vector space over a field, $\field$ if:
\begin{enumerate}[label=\roman*)]
  \item $\vectorspace$ is an abelian group w.r.t addition;
  \item $\forall\ \vect{v} \in \vectorspace$ \& $\lambda \in \field, \lambda\vect{v} \in \vectorspace$;
  \item $\lambda(\vect{u} + \vect{v}) = \lambda\vect{u} + \lambda\vect{v}$;
  \item $\lambda(\mu\vect{v}) = (\lambda\mu)\vect{v}$; and,
  \item $1.\vect{v} = \vect{v}$.
\end{enumerate}

\subtitle{Theorem 7.08 - }{Vector Spaces over Fields}
Let $W$ be a vector space over a field, $\field$, and $U$ be a set. Then define
$$F(U, W) := {f : U \to W}$$
$F(U, W)$ is a vector space over $\field$.\\
This means $F(U, W)$ is linear so $\forall\ \lambda \in \field\ \&\ f, g \in F(U, W)$ then
$$(f + g)(u) = f(u) + g(u),\quad (\lambda f)(u) = \lambda f(u)$$

\subsection{Subspace, Linear Combinations \& Span}

\subtitle{Definition 7.09 - }{Subspace}
Let $\vectorspace$ be a vector space over a field $\field$ and $W \subset \vectorspace$, $W$ is a subspace if it is a vector space for the operations inherited from $\vectorspace$.\\

\subtitle{Theorem 7.10 - }{Properties of Subspaces}
Let $\vectorspace$ be a vector space and $U \subset \vectorspace$ be a subspace, then $U$ has the following properties:
\begin{enumerate}[label=\roman*)]
  \item Not empty - $U \not = \emptyset$;
  \item Closed under addition - $\forall\ u, v \in U; (u + v) \in U$; and,
  \item Closed under multiplication - $\forall\ \lambda \in \field, u \in U; \lambda u \in U$.
\end{enumerate}

\subtitle{Theorem 7.11 - }{Subsets of Subspaces}
Let $\vectorspace$ be a vector space over $\field$ and $U, W \subset \vectorspace$ be subspaces.\\
Then $U \cap W$ is a subspace of $\vectorspace$.\\

\subtitle{Remark 7.12 - }{Linear Independence and Span}
Let $\vectorspace$ be a vector space over field, $\field$, and $S \subset \vectorspace$.\\
$S$ is linearly dependent if there exists $v \in \vectorspace$ such that $span(S) = span(S\backslash\{v\})$.\\

\subtitle{Definition 7.13 - }{Finite Dimensional}
Let $\vectorspace$ be a vector space over $\field$.\\
$\vectorspace$ is finitely dimensional if it is a span of a finite set, $S \subset \vectorspace$, of vectors.\\
\underline{N.B.} - If a vector space is not finite dimensional, then it is \textit{infinitely dimensional}.\\

\subtitle{Theorem 7.14}{}
Let $\vectorspace$ be a vector space over $\field$ with $\basis, U \subset \vectorspace$.\\
If $\basis$ is a basis for $\vectorspace$, with $|\basis| < \infty$, and $U$ is linearly independent then $$|U| \leq |\basis|$$

\subtitle{Theorem 7.15 - }{Linearly Independent Sets as Bases}
Let $\vectorspace$ be a vector space over $\field$ with $U \subset \vectorspace$ as a linearly independent set.\\
Then $U$ can be extended to form a basis of $\vectorspace$.

\subsection{Direct Sums}

\subtitle{Definition 7.16 - }{Direct Sum}
Let $\vectorspace$ be a vector space over $\field$ and $U, W \subset V$ be subspaces with $U \cap W = \emptyset$ then
$$U \oplus W := U + W$$
This is the \textit{direct sum} of $U$ and $W$.\\

\subtitle{Theorem 7.17 - }{Dimension of Direct Sum}
Let $\vectorspace$ be a vector space over $\field$ and $U, W \subset V$ be subspaces with $U \cap W = \emptyset$ then
$$dim(U \oplus W) = dim(U) + dim(W)$$

\subtitle{Theorem 7.18 - }{Complement}
Let $\vectorspace$ be a vector space over $\field$ and $U, W \subset V$ be subspaces with $U \cap W = \emptyset$ if
$$U \oplus W = V$$
then $W$ is said to be the complement of $U$ in $V$.

\subsection{Rank-Nullity Theorem}

\subtitle{Definition 7.19 - }{Rank \& Nullity}
Let $\vectorspace\ \&\ \mathbb{W}$ be vector spaces over $\field$ and $T : \vectorspace \to \mathbb{W}$ be a linear map. Then
$$rank(T) := Dim(Im(T)),\quad nullity(T) := Dim(Ker(T))$$

\subtitle{Theorem 7.20 - }{Rank-Nullity Theorem}
Let $\vectorspace\ \&\ \mathbb{W}$ be vector spaces over $\field$ and $T : \vectorspace \to \mathbb{W}$ be a linear map, with $dim(\vectorspace) < \infty$ then
$$Rank(T) + Im(T) = Dim(\vectorspace)$$

\subsection{Projection}

\subtitle{Defintion 7.21 - }{Projection}
A linear map $P : \vectorspace \to \vectorspace$ is called a projection if $P^2 = P$.\\

\subtitle{Theorem 7.22 - }{Image of Projection}
Let $P : V \to V$ be a projection then $v \in Im(P)$ iff $P(v) = v$.\\

\subtitle{Theorem 7.23 - }{Direct Sum of Projection}
Let $P : V \to V$ be a projection then
$$V = Ker(P) \oplus Im(P)$$

\subsection{Isomorphisms}

\subtitle{Definition 7.24 -}{Isomorphisms}
Let $\vectorspace\ \&\ \mathbb{W}$ be vector spaces over $\field$. \\
We say that the map $T : \vectorspace \to \mathbb{W}$ is an \textit{isomorphism} between $\vectorspace\ \&\ \mathbb{W}$ if
\begin{enumerate}[label=\roman*)]
  \item $T$ is linear; and,
  \item $T$ is bijective.
\end{enumerate}
\underline{N.B.} - If an \textit{isomorphism} exists between $V$ \& $W$, then they are said to be isomorphic.\\

\subtitle{Theorem 7.25 - }{Dimension of Isomorphic Spaces}
Let $V$ be a finitely dimensional vector space over $\field$.\\
If $W$ is isomorphic to $V$ then $$dim(V) = dim(W)$$
This definition can be extended to say\\
\centerline{\textit{If two vector spaces have the same dimension, then they are isomorphic.}}\\

\subtitle{Proposition 7.26 - }{Multiple Bases}
Let $A = \{\vect{a}_1 , \dots , \vect{a}_n\}$ and $B = \{\vect{b}_1 , \dots , \vect{b}_n\}$ be different bases for $V$.\\
Define $T_A : \field^n \to V$ and $T_B : \field^n \to V$ such that
$$T_A(x_1, \dots , x_n) = x_1.\vect{a}_1 + \dots + x_n.\vect{a}_n; \quad T_B(x_1, \dots , x_n) = x_1.\vect{b}_1 + \dots + x_n.\vect{b}_n$$
Then for all $\vect{v} \in V$ there are two ways of expressing $\vect{v}$.
$$x_1.\vect{a}_1 + \dots + x_n.\vect{a}_n = \vect{v} = x_1.\vect{b}_1 + \dots + x_n.\vect{b}_n$$
Unless $A = B$ then $x_i \not = y_i$ for at least one $i \in \nat, i \leq n$.\\

\subtitle{Theorem 7.27 - }{Conversion Matrices}
Let $A\ \&\ B$ be different bases for vector space $V$, with $dim(V) = n$.\\
Then an $n \times n$ matrix, $C_{AB}$, can be used to convert elements given in basis $A$ to now be givin in basis $B$.\\
Let $\vect{v} \in V$ and $\vect{x} = T_{A}(\vect{x})\ \& \vect{y} = T_{B}(\vect{x})$ then $$\vect{y} = C_{AB}\vect{x}$$

\subtitle{Theorem 7.28 - }{General Relationship between Bases}
Let $V$ be a vector space over $\field$, with $dim(V) = n$.\\
Let $A\ \&\ B$ be different bases for $V$ with $A = \{\vect{a}_1 , \dots , \vect{a}_n\}\ \&\ B = \{\vect{b}_1 , \dots , \vect{b}_n\}$.\\
Then $\forall\ \vect{v} \in V$ we have that $$\vect{v} = \sum_{i=1}^{n} v_i . \vect{a}_i = \sum_{i=1}^{n} v_i . \vect{b}_i $$
Let $C_{AB} = (c_{ij})$ be the conversion matrix from $A$ to $B$ then $$v_j = \sum_{i=1}^{n} c_{ij}\vect{b_i}$$

\subtitle{Theorem 7.29 - }{Properties of Transition Matrices}
Let $A, B, C \subset V$ all be different bases for $V$. Then
\begin{enumerate}[label=\roman*)]
  \item $C_{AA} = I$;
  \item $C_{AB}C_{BA} = I$; and,
  \item $C_{CA}C_{AB} = C_{CB}$.\\
\end{enumerate}

\subtitle{Theorem 7.30 - }{Linear Maps between Vector Spaces as Matrices}
Let $\vectorspace\ \&\ \mathbb{W}$ both be vector spaces over $\field$, with $dim(\vectorspace) = n$ and $dim(\mathbb) = m$, and $T : \vectorspace \to \mathbb{W}$ be a linear map.\\
Let $A = \{\vect{a}_1, \dots , \vect{a}_n\} \subset \vectorspace$ and $B = \{\vect{b}_1, \dots , \vect{b}_n\} \subset \mathbb{W}$ be bases for $\vectorspace\ \&\ \mathbb{W}$ respectively.
Then we can define an $n \times m$ matrix $$M_{AB}(T) = (m_{ij}) \in M_{n,m}(\field)$$
Where $m_{ij}$ are defined to satisfy
$$T(a_j) = \sum_{i=1}^{m} m_{ij}b_i$$
Then
$$\vect{w} = M_{AB}(T)\vect{v}$$
For $\vect{v} \in \vectorspace\ \&\ \vect{w} \in \mathbb{W}$.\\

\subtitle{Theorem 7.31 - }{Change Basis of Linear Map}
Let $V$ be a vector space over $F$ and $U, W \subset V$ be different bases for $V$.\\
Define $T : V \to V$ be a linear map and $C$ to be the transition matrix from basis $U \to W$.\\
Then $C^{-1}$ is the transition matrix from $W \to U$.\\
Set $A$ to be the matrix representation of $T$ in basis $U$. Then
$$A' = C^{-1}AC$$
Where $A'$ is the matrix representation of $T$ in basis $W$.

\section{Eigenvalues \& Eigenvectors}

\subsection{Characteristic Polynomial}

\subtitle{Definition 8.01 - }{Eigenvectors \& Eigenvalues}
Let $\vect{v} \in V\backslash\{\vect{0}\}$ and $T : V \to V$ be a linear operator.\\
$\vect{v}$ is called an \textit{eigenvector} of $T$ if $$T(\vect{v}) = \lambda\vect{v}, \quad \lambda \in \field$$
This $\lambda$ is the associated \textit{eigenvalue} for $\vect{v}$.\\

\subtitle{Definition 8.02 - }{Spectrum}
The set of eigenvectors of a linear operator $T : V \to V$ is called the \textit{spectrum} of $T$, generally denoted as $$Spec(T) := \{\vect{v} \in V : T(\vect{v}) = \lambda\vect{v}, \lambda \in \field\}$$

\subtitle{Defintion 8.03 - }{Diagonisable}
A linear operator is \textit{diagonisable} if there exists a basis of eigenvectors for it.\\

\subtitle{Remark 8.04 - }{Finding Eigenvalues}
Let $A$ be the matrix which represents a linear operator $T$, and $\vect{x}$ be a general eigenvector for $T$
\begin{alignat*}{1}
  T(\vect{x}) = A\vect{x} &= \lambda\vect{x}
  => (A - \lambda.I)\vect{x} = \vect{0}
\end{alignat*}
Then $\lambda$ is an eigenvalue if it satisfies $$det(A - \lambda.I) = 0$$

\subtitle{Definition 8.05 - }{Characteristic Polynomial}
The polynomial which is equivalent to $det(A - \lambda.I)$ is called the \textit{characteristic polynomial} of $A$.
$$p_A(\lambda) := det(A - \lambda.I)$$
\underline{N.B.} - $\lambda$ is an eigenvalue for $A$ if $p_A(\lambda) = 0$\\

\subtitle{Definition 8.06 - }{Eigenspace}
Let $\lambda \in \field$ be an eigenvalue of $T$, then the corresponding \textit{eigenspace} is defined as
$$V_\lambda := ker(T - \lambda.I)$$

\subtitle{Remark 8.07 - }{Finding Eigenvectors}
Once we have found all $\lambda_1 , \dots , \lambda_k$ that satisfy $p_A(\lambda_i) = 0$ then we can find the eigenvectors, $\vect{x}_i$, of $A$
$$(A - \lambda.I)\vect{x}_i = \vect{0}$$
A good way to start is to produce the linear equations
$$\sum_{j=1}^{n} (A - \lambda.I)_{ij}.x_j = 0$$
For all $i \leq n$. Then solve these, as a series of simultaneous equations, to find the values $x_j$ which produce the eigenvector $\vect{x}$.\\
Repeat this process for all $\lambda_1 , \dots , \lambda_k$ to find all eigenvectors for $A$.\\

\subtitle{Theorem 8.08 - }{Similar Characteristic Polynomial}
Let $C$ be an invertible matrix.\\
Define $A' = C^{-1}AV$ where $A$ \& $A'$ are conjugate or similar.\\
Then $p_A(\lambda) = p_{A'}(\lambda)$.\\

\subtitle{Theorem 8.09 - }{Characteristic Polynomial \& Basis}
The characteristic polynomial for $T$ is the same, regardless of the basis of $T$.\\

\subtitle{Definition 8.10 - }{Trace}
Let $A \in M_n(\field)$.\\
Then the \textit{trace} of $A$ is defined as
$$Tr(A) := \sum_{i=1}^{n}a_{ii}$$
\underline{N.B.} - \textit{Trace} is sometimes called \textit{Spur}.\\

\subtitle{Remark 8.11}{}
As the terms after the first term of the determinat of a matrix do not contribute to the powers of $\lambda$ in the characteristic equation then we can ignore them.
This means we can deduce that
$$p_T(\lambda) = (-1)^n\lambda^n + (-1)^{n-1}(Tr(A)) + \dots + det(A)$$

\subtitle{Theorem 8.12 - }{Diagonalised Matrix}
Let $T$ be a \textit{diagonisable} matrix with eigenvalues $\{\lambda_1, \lambda_2, \dots , \lambda_n\}$.\\
Then $T$ can be represented by $$\Delta = \begin{pmatrix} \lambda_1 & 0  & \dots & 0\\ 0 & \lambda_2 & & \vdots \\ \vdots & & \ddots & \vdots \\ 0 & \dots & \dots & \lambda_n \end{pmatrix}$$
\underline{N.B.} - $T$ can also be represented in any basis with, $C$ as the transition matrix, by $C^{-1}\Delta C$.\\

\subtitle{Theorem 8.13 - }{Relationship between Matrix and its Diagonalised Form}
Let $T$ be a matrix and $\Delta$ be its diagonalised form, then
$$Det(T) = Det(\Delta) = \prod_{j=1}^n \lambda_j$$
And $$Tr(T) = Tr(\Delta) = \sum_{j=1}^n \lambda_j$$

\subtitle{Theorem 8.14 - }{Distinct Eigenvectors and Diagonisability}
Eigenvectors, which correspond to distinct eigen values, are linearly independent.\\
Thus if a matrix, $A$, has only distinct eigenvalues then it is diagonisable.

\subsection{Roots of Characteristic Polynomial}

\subtitle{Remark 8.15 - }{Degree of Characteristic Equation}
Eigenvalues are roots of $p_A(\lambda) = 0$ where $p_A$ is an equation of degree $dim(A)$.\\

\subtitle{Remark 8.16 - }{Non-Distinct Roots of Characteristic Equation}
If the roots of $P_A(\lambda)$ are not distinct then $A$ may be diagonisable depending on how many eigenvectors are found.\\

\subtitle{Theorem 8.17 - }{Vieta's Theorem}
If $\lambda_1 , \dots , \lambda_n$ are roots of the Polynomial
$$\lambda_n + a_1\lambda^{n-1} + \dots + a_n = 0 \equiv p(\lambda) = 0$$
Then $p(\lambda) = (\lambda-\lambda_1)(\lambda - \lambda_2) \dots (\lambda - \lambda_n)$.\\
So $p(\lambda)$ factorises in the product $\prod_{i=1}^{n}(\lambda - \lambda_i)$ but the $\lambda_i$s are not necessarily distinct.\\

\subtitle{Definition 8.18 - }{Multiplicity of Roots}
Let $\lambda_1 \in \complex$ of characteristic polynomial, $p(\lambda)$.\\
$\lambda_1$ has multiplicity $m_1 \in \nat$ if
$$p(\lambda_1) = \frac{dp}{d\lambda}(\lambda_1) = \dots = \frac{d^{m_1 - 1}p}{d\lambda^{m_1 - 1}}(\lambda_1) = 0$$
This means that $(\lambda - \lambda_1)^{m_1}$ is a factor of $p(\lambda)$.\\

\subtitle{Definition 8.19 - }{Geometric \& Algebraic Multiplicity}
Let $\lambda \in spec(T)$ and $V_\lambda$ be the corresponding eigenspace.
\begin{enumerate}[label=\roman*)]
  \item $\lambda$ has \textit{geometric multiplicity}, $m_g(\lambda) \in \nat$, if $dim(V_\lambda) = m_g(\lambda)$; and,
  \item $\lambda$ has \textit{algebraic multiplicity}, $m_a(\lambda) \in \nat$, if $\lambda$ has multiplicity $m_a$ of $p_T(\lambda)$.\\
\end{enumerate}

\subtitle{Theorem 8.20 - }{Relationship between Geometric \& Algebraic Multiplicity}
Let $\lambda \in spec(T)$ then $$m_g(\lambda) \leq m_a(\lambda)$$

\subtitle{Theorem 8.21 - }{}
Let $T$ be a linear operator on an $n$ dimensional space over $\complex$ or $\real$, with eigenvalues $\lambda_1 , \dots , \lambda_n$, which are not necessarily distinct. Then
$$det(T) = \prod_{i=1}^n \lambda_i \quad \& \quad tr(T) = \sum_{i=1}^n \lambda_i$$

\section{Inner Product Spaces}

\subsection{Inner Product, Norm \& Orthogonality}

\subtitle{Definition 9.01 - }{Inner Product (Complex)}
Let $V$ be a vector space over $\complex$.\\
An \textit{inner product} on $V$ is a map, $\langle V, V \rangle : V \times V \to \complex$, with the following properties:
\begin{enumerate}[label=\roman*)]
  \item $\langle v, v \rangle \geq 0$;
  \item $\langle v, w \rangle = \overline{\langle v, w \rangle}$;
  \item $\langle u, v + w \rangle = \langle u, v \rangle + \langle u, w \rangle$; and,
  \item $\langle \lambda.u , v \rangle = \lambda \langle u, v \rangle$.
\end{enumerate}
Where $u, v, w \in V$ and $\lambda \in \complex$.\\

\subtitle{Definition 9.02 - }{Inner Product (Real)}
Let $V$ be a vector space over $\real$.\\
An \textit{inner product} on $V$ is a map, $\langle , \rangle : V \times V \to \complex$, with the following properties:
\begin{enumerate}[label=\roman*)]
  \item $\langle v, v \rangle \geq 0$;
  \item $\langle v, w \rangle = \langle w, v \rangle$;
  \item $\langle u, v + w \rangle = \langle u, v \rangle + \langle u, w \rangle$; and,
  \item $\langle \lambda.u , v \rangle = \lambda \langle u, v \rangle$.
\end{enumerate}
Where $u, v, w \in V$ and $\lambda \in \complex$.\\

\subtitle{Definition 9.03 - }{Inner Product Space}
Let $\vectorspace$ be a vector space with $\langle , \rangle$ as a defined inner product are called an \textit{inner product space}, denoted by
$$(\vectorspace, \langle , \rangle)$$
\underline{N.B.} - If $V$ is over $\complex$ then this is called a \textit{complex inner product space}. If $V$ is over $\real$ then this is called a \textit{real inner product space}.\\

\subtitle{Definition 9.04 - }{Norm}
Let $( V , \langle , \rangle )$ be an inner product space, then we define the associated norm as
$$||v|| := \sqrt{\langle v, v \rangle},\quad v \in V$$

\subtitle{Definition 9.05 - }{Orthogonal}
Let $(V , \langle , \rangle)$ be an inner product space, then
\begin{enumerate}[label=\roman*)]
  \item $v, w \in V$ are \textit{orthogonal}, $v \perp w$, if $\langle v, w \rangle = 0$; and,
  \item $U, W \subset V$ are \textit{orthogonal}, $U \perp W$, if $u \perp w\ \forall\ u \in U\ \&\ v \in V$.
\end{enumerate}

\subtitle{Definition 9.06 - }{Orthogonal Complement}
Let $(V, \langle , \rangle)$ be an inner product space and $W \subset V$.\\
The \textit{orthogonal complement} is defined as
$$W^\perp := \{ v \in V : v \perp w\ \forall\ w \in W \}$$

\subtitle{Theorem 9.07 - }{Norm of Orthogonal Elements}
Let $(V, \langle , \rangle)$ be an inner product space and $v, w \in V$ with $v \perp w$, then
$$||v + w||^2 = ||v||^2 + ||w||^2$$

\subtitle{Definition 9.08 - }{Orthonormal Basis}
Let $(V, \langle , \rangle)$ be an inner product space.\\
A basis, $\basis = \{ v_1 , \dots , v_n\}$, is called an \textit{orthonormal basis} if
$$\langle v_i , v_j \rangle = \delta_{ij} := \begin{cases} 1 \quad i = j;\\ 0 \quad i \not = j. \end{cases}$$

\subtitle{Theorem 9.09 - }{Properties of Orthogonal Basis}
Let $(V, \langle , \rangle)$ be an inner product space and $\basis = \{ v_1 , \dots , v_n\}$ an orthonormal basis.\\
Then $\forall\ v, w \in V$,
\begin{enumerate}[label=\roman*)]
  \item $v = \sum_{i=1}^n \langle v_i , v \rangle v_i$;
  \item $\langle v , w \rangle = \sum_{i=1}^n \overline{\langle v_i, v \rangle}\langle v_i, w \rangle$; and,
  \item $||v|| = \left[\sum_{i=1}^n |\langle v_i, v \rangle|^2\right]^{1/2} $.
\end{enumerate}

\subsection{Construction of Orthonormal Basis}

\end{document}
