% [fontSize]{documentType}
\documentclass[11pt]{article}

\usepackage[margin=1in, paperwidth=8.3in, paperheight=11.7in]{geometry} % Define Margins
\usepackage{amsfonts}  % Adds symbols for natural numbers etc.
\usepackage{amsmath}
%Body of document goes between \begin{document} & \end{document}
\begin{document}

% Easily create a title for documents
\title{Hello World! Hello \LaTeX !}
\author{Dominic Hutchinson}
\date{\today} % Automatically adds today's date when compiled
\maketitle

\tableofcontents

\newpage

% Just typing text leaves it as standard text
\section{Text}
\subsection{General}
Hello World!\newline

% New lines are considered as new paragraphs
% \\ starts a new line but in the same paragraph
Paragraphs are useful as they make reading large blocks of text easier.\\
It really is great.\newline

% Can format text
\subsection{Formatting}
This text with be \textit{italised}.\\
This text will be \textbf{bold}.\\
This text will be \textsc{small caps}.\\
This text will be \texttt{typewriter typeface}.\\
This text will be \begin{tiny} tiny \end{tiny}.\\
This text will be \begin{small} small \end{small}.\\
This text will be normal.\\
This text will be \begin{large} large \end{large}.\\
This text will be \begin{Large} Large \end{Large}.\\
This text will be \begin{huge} huge \end{huge}.\\
This text will be \begin{Huge} Huge \end{Huge}.\\

\subsection{Alignment}
\begin{center} This text is centre aligned. \end{center}
\begin{flushright} This text is right aligned. \end{flushright}
\begin{flushleft} This text is left aligned. \end{flushleft}

\subsection{Set Notation}
Natural Numbers:\ $\mathbb{N}$\\
Integers Numbers:\ $\mathbb{Z}$\\
Real Numbers:\ $\mathbb{R}$\\
Complex Numbers:\ $\mathbb{C}$\\
% For mathematical notation we need to use special notation
% Write inline equations between $s.
\newpage
\section{Maths Notation}
\subsection{General}
Suppose we are given a rectangle with sides of length $(x+1)$ and $(x+3)$. The equation $A=x^2+4x+3$ represents the area of the rectangle.\newline

% To have equations on seperate lines to text, use $$
Suppose we are given a rectangle with sides of length $(x+1)$ and $(x+3)$. The equation $$A=x^2+4x+3$$ represents the area of the rectangle.\newline

% Sometimes in line equations are shrunk in order to fit inline
% use \displaystyle{} to make the equations standard size
About $\displaystyle{\frac{2}{3}}$ of the glass is full.\newline

% Some common maths notation
\subsection{Superscript}
$$2x^3$$
$$2x^{34}$$
$$2x^{2x^4+5}$$

\subsection{Subscript}
$$x_1$$
$$x_1, ..., x_{10}$$
$$x_{1_2}$$

\subsection{Greek Letters}
$$\alpha, \beta, ..., \omega$$
$$A=\pi r^2$$

\subsection{Trigonometry}
$$1=\sin^2{x} + \cos^2{x}$$

\subsection{Logs}
$$\log{x}, log_{10}{x}, log_2{x}$$
$$\ln{x}$$

\subsection{Roots}
$$\sqrt{4}=\pm2$$
$$\sqrt[3]{8}=2$$
$$z=\sqrt{x^2+y^2}$$

\subsection{Fractions}

About $\frac{2}{3}$ of the glass is full.\newline

\subsection{Brackets}
$$(x+1)$$
$$3[2+(x+1)]$$
% Need to but \ in front of {, }, $ etc.
$$\{a,b,c\}$$
% Brackets don't automatically resize for fractions
$$3\left(\frac{2} {5} \right)$$
$$3\left[\frac{2} {5} \right]$$
$$3\left\{\frac{2} {5} \right\}$$
$$\left|\frac{x}{x+1} \right|$$
% You can miss off a bracket by using a '.' instead of a bracket
$$\left. \frac{dy}{dx} \right|_{x=1}$$

\subsection{Tables}
% {alignment & #cols}. Enter one letter per col
% c = centre, 
\begin{tabular}{|c|cc|}

% & between columns, \\ starts next row
\hline
$x$ & $1$ & $2$ \\ \hline
$f(x)$ & 11 & 12 \\ \hline

\end{tabular} \newline

% Automatically in maths mode
% Lines up equations, right-justify by default
	% Placing an & around a character tells compiler to align by said character
% Numbers equations by default
	% Placing an * after 'eqnarray' removes numbers
\subsection{Equation Arrays}
\begin{eqnarray*}
5x^2-9&=&x^2+3\\
4x^2&=&12\\
x^2&=&3\\
x&=&\pm\sqrt{3}
\end{eqnarray*}

\subsection{Calculus}
The equation $f(x)=(x-3)^2+\frac{1}{2}$ with domain $\mathrm{D}_f:(-\infty,\infty)$ has range $\mathrm{R}_f:\left[-\frac{1}{2}, \infty\right]$\\

$\displaystyle{\lim \limits_{x \to a} \frac{f(x)-f(a)}{x-a} =f'(a)}$\\

$\displaystyle{\int \sin (x)\ dx = -\cos (x)+c}$\\

$\displaystyle{\int_a^b \sin (x)\ dx = \cos (a) - \cos (b)}$\\

$\displaystyle{\sum \limits_{n=1}^{\infty}ar^n=a+ar+ar^2+ \cdots +ar^n}$

\subsection{Matrices}
\begin{eqnarray*}
\begin{pmatrix}
1 & 2 & 3 \\
4 & 5 & 6 \\
7 & 8 & 9
\end{pmatrix}\\
\begin{pmatrix}
1 \\
2 \\
3 
\end{pmatrix} \\
\begin{bmatrix}
1 & 2 & 3 \\
4 & 5 & 6 \\
7 & 8 & 9
\end{bmatrix}
\end{eqnarray*}


\end{document}
\grid
