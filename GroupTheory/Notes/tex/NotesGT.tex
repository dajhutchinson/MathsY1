% Ctrl + alt + b to build & preview (Linux)
\documentclass[11pt,a4paper]{article}

\usepackage[margin=1in, paperwidth=8.3in, paperheight=11.7in]{geometry}
\usepackage{amsfonts}
\usepackage{amsmath}
\usepackage{enumerate}
\usepackage{enumitem}
\usepackage{fancyhdr}

\begin{document}

\pagestyle{fancy}
\setlength\parindent{0pt}
\allowdisplaybreaks

\renewcommand{\headrulewidth}{0pt}
\newcommand{\vect}[1]{\boldsymbol{#1}}
\newcommand{\subtitle}[2]{\textbf{#1}\textit{#2}\\}
\newcommand{\dotprod}[0]{\boldsymbol{\cdot}}
\newcommand{\real}[0]{\mathbb{R}}
\newcommand{\nat}[0]{\mathbb{N}}
\newcommand{\field}[0]{\mathbb{F}}
\newcommand{\integers}[0]{\mathbb{Z}}
\newcommand{\vectorspace}[0]{\mathbb{V}}
\newcommand{\basis}[0]{\mathbb{B}}

% Cover page title
\title{Introduction to Group Theory - Notes}
\author{Dom Hutchinson}
\date{\today}
\maketitle

% Header
\fancyhead[L]{Dom Hutchinson}
\fancyhead[C]{Introduction to Group Theory - Notes}
\fancyhead[R]{\today}

\tableofcontents

% Start of content
\newpage

\section{Symmetries}

\subtitle{Definition 1.01 - }{Permutation}
A \textit{permutation} of a set, $G$, is a bijection of the form $f : G \to G$.\\
\underline{N.B.} - Since the composition of two bijections is also a bijection, then the composition of two permutations is a permutation.\\

\subtitle{Definition 1.02 - }{Symmetries of a Polygon}
A \textit{symmetry} of an n-sided polygon is a permutation of the vertices which preserves adjacency.\\
So if the vertices $u\ \&\ v$ are adjacent then the permutation $f$ is a symmetry if $f(u)$ \& $f(v)$ are adjacent.\\

\subtitle{Remark 1.03 - }{Symmetries}
When dealing with symmetries of a shape then they can only be rotations or reflections.\\

\subtitle{Definition 1.04 - }{Identity}
The trivial symmetry, which maps an element to itself, is known as the identity.\\

\subtitle{Remark 1.05 - }{Composition of Permutations}
Let $R, S\ \&\ T$ be permutations.\\
Then $(RS)T$ means do $T$, then $S$, then $R$. So $$(RS)T = R(ST)$$

\subtitle{Remark 1.06 - }{One-Line Notation}
Let $S = \{a_1, \dots, a_n\}$ be a set and $\sigma : S \to S$ be a permutation.\\
\textit{One-Line notation} denotes the result of $\sigma$ by $$\begin{pmatrix} \sigma(a_1) & \dots & \sigma(a_n) \end{pmatrix}$$
So if $\sigma$ maps $1 \to 2, 2 \to 3, \dots, n \to 1$ then it can be denoted by $$\begin{pmatrix} 2  & 3 & \dots & n & 1 \end{pmatrix}$$

\subtitle{Remark 1.07 - }{Two-Line Notation}
Let $S = \{a_1, \dots, a_n\}$ be a set and $\sigma : S \to S$ be a permutation.\\
\textit{Two-Line notation} denotes the result of $\sigma$ by $$\begin{pmatrix} a_1 & \dots & a_n \\ \sigma(a_1) & \dots & \sigma(a_n) \end{pmatrix}$$
So if $\sigma$ maps $1 \to 2, 2 \to 3, \dots, n \to 1$ then it can be denoted by $$\begin{pmatrix} 1 & 2 & \dots & n-1 & n \\ 2  & 3 & \dots & n & 1 \end{pmatrix}$$

\subtitle{Remark 1.08 - }{Cycle Decomposition Notation}
Let $S = \{a_1, \dots, a_n\}$ be a set and $\sigma : S \to S$ be a permutation.\\
\textit{Cycle Decomposition Notation} denotes $\sigma$ as the product of disjoint cycles.\\
Each element in a cycle goes the position of the element after it in the list, the last element goes to the position of the first.\\
$()$ denotes no variation. The operation of $\sigma$ is denoted by $$\begin{pmatrix} a_1 & \sigma(a_1) & \sigma(\sigma(a_1)) & \dots & \sigma(\dots\sigma(a_1)\dots)\end{pmatrix}$$

\section{Groups}

\subtitle{Definition 2.01 - }{Binary Operation}
A \textit{binary operation} on a set $X$ is a function of the form $f : X \times X \to X$.\\

\subtitle{Remark 2.02 - }{Asteriks Notation}
Binary operations are general denoted by an $*$. $$f(x, y) = x*y$$

\subtitle{Remark 2.03 - }{Multiplicty Notation}
Multiplicity notation is used to simplify equations with a single binary operator, by not writting $*$. $$x * y = xy$$

\subtitle{Remark 2.04 - }{Set of Permutations}
A set of permutations have a binary operation for composition.\\
Let $f, g, h$ be permutations of a set $X$ and $x \in X$ $$f(x) \times g(x) \to h(x)$$

\subtitle{Definition 2.05 - }{Commutativity}
A binary operation, $*$, on a set $X$ is \textit{commutative} if order of input doesn't affected the outcome. $$x * y = y * x,\forall\ x, y \in X$$

\subtitle{Definition 2.06 - }{Commute}
If $x, y \in G$ satisfy $x*y = y*x$ then it is said that $x\ \&\ y$ commute.\\

\subtitle{Defintion 2.07 - }{Group}
A group is a set, $G$, with an associated binary operation, $*$, that
\begin{enumerate}[label=\roman*)]
  \item Is \textit{associative}, $(x * y) * z = x * (y * z)\ \forall\ x, y, z \in G$;
  \item Has as an \textit{identity element}, $\exists\ e \in G$ such that $x * e = x = e * x$; and,
  \item Has an inverse element $\forall\ x \in G\ \exists\ x^{-1} \in G$ st $xx^{-1} = e = x^{-1}x$.\\
\end{enumerate}

\subtitle{Remark 2.08 - }{Group Notation}
The group of set $G$ and binary operation $*$ is denoted by $(G, *)$.\\

\subtitle{Definition 2.09 - }{Abelian Group}
An \textit{Abelian group}, $(G, *)$ is one where $*$ is commutative.

\section{Elementary Consequences of the Definition}

\subtitle{Proposition 3.01 - }{Right Cancellation}
If $a, b, x \in G$ and $ax = bx$ then $a = b$.\\

\subtitle{Proposition 3.02 - }{Left Cancellation}
If $a, b, x \in G$ and $x => $ then $a = xba = b$.\\

\subtitle{Proposition 3.03 - }{Uniqueness of Identity}
If $a, x, e \in G$ with $e$ as the identity of $G$ then $$ax = a => e = x$$

\subtitle{Proposition 3.04 - }{Uniqueness of Inverses}
If $x, y, e \in G$ with $e$ as the identity of $G$ then $$xy = e => x = y^{-1}\ \&\ y = x^{-1}$$

\subtitle{Proposition 3.05 - }{Inverse of Inverse}
Let $x \in G$ then $$(x^{-1})^{-1} = x$$

\subtitle{Proposition 3.06 - }{Composite Inverses}
Let $x, y \in G$ then $$(xy)^{-1} = y^{-1}x^{-1}$$

\subtitle{Definition 3.07 - }{Caley Table}
Let $e, x, y$ be all the elements of $G$ then the result of all compositions can be displayed in a \textit{Caley Table}.
\begin{center}
\begin{tabular}{c|ccc}
 & e & x & y \\
 \hline
 e & e & x & y \\
 x & x & xx & yx \\
 y & y & xy & yy
\end{tabular}
\end{center}
The operation of the column is done first, then the operation of the row.\\
\underline{N.B.} - All values in any given column or row are unique, so all elements of $G$ appear exactly once.\\

\subtitle{Definition 3.08 - }{Powers of Elements}
If $n > 0$ then $x^n$ means $x * \dots * x$ $n$ times. $$x^{-n} = (x^{n})^{-1} = (x^{-1})^n,\quad x^0 = e$$

\subtitle{Definition 3.09 - }{Composition of Powers}
For $m,n \in \integers$ $$x^mx^n = x^{m+n}$$

\section{Dihedral Groups}

\subtitle{Definition 4.01 - }{Order}
The \textit{order} of a group $G$ is the number of elements in $G$.\\
\underline{N.B.} - Order of $G$ is denoted by $|G|$.\\

\subtitle{Definition 4.02 - }{Dihedral Groups}
The \textit{dihedral group} $D_{2n}$ is the group of symmetries of a regular n-sided polygon, with $n \geq 3$.
\underline{N.B.} - $|D_{2n}| = 2n$.\\

\subtitle{Proposition 4.03 - }{Elements of Dihedral Group}
Let $a$ describe a rotation by $\frac{2\pi}{n}$ and $b$ a reflection then
$$D_{2n} = \{ e, a, a^2, \dots, a^{n-1}, b, ab , \dots , a^{n-1}b\}$$
\underline{N.B.} - $a^n = e = b^2,\quad a^{-1} = a^{n-1}, b = b^{-1}$.\\

\subtitle{Proposition 4.04 - }{Reflections \& Rotations}
Let $a$ denote a rotation and $b$ denoted a reflection then $$ab = ba^{-1}$$

\section{Subgroups}

\subtitle{Definition 5.01 - }{Subgroup}
A \textit{subgroup} of a group $G$ is a group formed of a subset of $G$ with the same associated operation.\\
\underline{N.B.} - $H$ being a subgroup of $G$ is denoted by $H \leq G$.\\

\subtitle{Definition 5.02 - }{Non-Trivial Subgroup}
A subgroup $H$ of $G$ is non-trivial if $H \not = \{ e \}$.\\

\subtitle{Definition 5.03 - }{Proper Subgroup}
A subgroup $H$ of $G$ is a \textit{proper subgroup} if $H \not = G$.\\

\subtitle{Theorem 5.04 - }{Subgroup}
A subset $H$ of a group $G$ is a subgroup iff
\begin{enumerate}[label=\roman*)]
  \item It is closed under the binary operation $x, y \in H => xy \in H$;
  \item It has an identity element $\exists\ e \in H$ st $xe = x\ \forall\ x \in H$; and,
  \item All elements have an inverse $\forall\ x \in H\ \exists\ x^{-1} \in H$ st $xx^{-1} = e$.\\
\end{enumerate}

\subtitle{Proposition 5.05 - }{Pairs of Subgroups}
Let $G, H\ \&\ K$ be groups with $H \leq G$ \& $K \leq G$ then $H \cap K \leq G$.

\section{Order of Elements}

\subtitle{Definition 6.01 - }{Order of an Element}
Let $x \in G$ such that $x^n = e$, then the \textit{order} of $x$ is the smallest such $n$.$$ord(x) = n$$
\underline{N.B.} - If there is no such $n$ then $ord(x) = \infty$.\\

\subtitle{Proposition 6.02 - }{Uniqueness of Powers}
Let $x \in G$ with $ord(x) = \infty$ then $$x^i \not = x^j\ \forall\ i \not = j$$

\subtitle{Theorem 6.03 - }{Order Elements in a Finite Group}
Every element of a finite group has finite order.\\

\subtitle{Theorem 6.04 - }{Properties of Order of an Element}
Let $x \in G$ such that $ord(x) = n < \infty$ then if
\begin{enumerate}[label=\roman*)]
  \item $x^i = e \iff n | i$;
  \item $x^i = x^j \iff i \equiv j (mod\ n)$;
  \item $x^{-1} = x^{n-1}$; and,
  \item The powers of $x$ less than $n$ are all distinct.
\end{enumerate}

\subtitle{Proposition 6.05 - }{Order of Powers of Elements}
Let $x \in G, i \in \integers$.
\begin{enumerate}[label=\roman*)]
  \item If $ord(x) = \infty$ then $ord(x^i) = \infty$ if $i \not = 0$; and,
  \item If $ord(x) = n < \infty$ then $\displaystyle{ord(x^i) = \frac{n}{gcd(n,i)}}$.
\end{enumerate}

\section{Cyclic Groups \& Cyclic Subgroups}

\subtitle{Definition 7.01 - }{Generating Cyclic Groups}
Let $G$ be a group and $x \in G$.\\
We define a \textit{cyclic group} generated by $x$ $$\langle x \rangle = \{x^i : i \in \integers \} \leq G$$.

\subtitle{Theorem 7.02 - }{Cyclic Subgroup}
Let $x \in G$ then $\langle x \rangle$ is a subgroup of $G$.\\

\subtitle{Definition 7.03 - }{Cyclic Group}
A group $G$ is cyclic if $G = \langle x \rangle$ for some $x \in G$.\\
\underline{N.B.} - Here $x$ is called the \textit{generator} of $G$.\\

\subtitle{Theorem 7.04 - }{Abelian Cyclic Groups}
Every cyclic group is abelian.\\

\subtitle{Theorem 7.05 - }{Finding Cyclic Groups}
Let $G$ be a group with $|G| = n < \infty$.\\
$G$ is cyclic iff $\exists\ x \in G$ such that $ord(x) = n$.\\

\subtitle{Theorem 7.06 - }{Subgroups of Cyclic Groups}
Every subgroup of a cyclic group is also a cyclic group.

\section{Groups from Modular Arithmetic}

\subtitle{Definition 8.01 - }{Congurence Class}
Let $n \in \nat$ then $a \equiv b (mod\ n)$ means $n | a - b$.\\
There are $n$ \textit{congurence classes} $[0], [1], \dots , [n-1]$ where every integer is in exactly one of these classes.
$$[x] = \{y \in \integers : x \equiv y (mod\ n) = \{ \dots , a-n , a , a+n , a+2n , \dots\}$$

\subtitle{Definition 8.02 - }{Congurence groups}
Let $n \in \nat$ then we denoted a congurence group of $n$ by
$$\frac{\integers}{n\integers} = {[0], [1] , \dots , [n-2] , [n-1]}$$
\underline{N.B} - Addition and multiplication are valid binary operations for congurence groups.\\

\subtitle{Definition 8.03 - }{Properties of Congurence Groups}
Let $[a], [b] \in \frac{\integers}{n\integers}$ for some $n \in \nat$ then
$$[a] + [b] = [a + b],\quad [a].[b] = [a.b]$$

\subtitle{Theorem 8.04 - }{Abelian Congurence Groups}
Let $n \in \nat$ then $\frac{\integers}{n\integers}$ is an abelian group.\\

\subtitle{Theorem 8.05 - }{Cyclic Abelian Congurence Groups}
The group $\left( \frac{\integers}{n\integers}, + \right) = \langle [1] \rangle$, so it is a cyclic group.\\
The group $\left( \frac{\integers}{n\integers}, \dotprod \right)$ is never a group for $n > 1$ as $[0][x] = [0] \not = [1] = e$.\\

\subtitle{Theorem 8.06 - }{Multiplicative Inverse of Congurence Groups}
$[a] \in \frac{\integers}{n\integers}$ has a multiplicative inverse if, and only if, $gcd(a, n) = 1$.\\

\subtitle{Definition 8.07 - }{Subset of $\frac{\integers}{n\integers}$ with multiplicative inverses}
$U_n$ is the subset of $\frac{\integers}{n\integers}$ such that
$$U_n = \left\{ [a] \in \frac{\integers}{n\integers}; gcd(a, n) = 1 \right\}$$
\underline{N.B.} - $(U_n, \dotprod)$ is an abelian group.

\section{Isomorphic Groups}

\subtitle{Definition 9.01 - }{Isomorphism}
Let $(G, *)$ and $(H, \dotprod)$ be groups.\\
An \textit{isomorphism} from $G$ to $H$ is a bijective function $\phi : G \to H$ such that $$\phi(x * y) = \phi(x) \dotprod \phi(y),\quad \forall\ x, y \in G$$
\underline{N.B.} - Since $\phi$ is bijective then there exists an inverse such that $\phi^{-1} : H \to G$.\\

\subtitle{Definition 9.02 - }{Isomorphic}
Let $G$ and $H$ be groups.\\
$G$ and $H$ are said to be \textit{isomorphic} if there exists an isomorphism $\phi : G \to H$. This is denoted by $G \cong H$.\\

\subtitle{Proposition 9.03 - }{Transitive property of Isomorphisms}
Let $G, H$ and $I$ be groups.\\
If $G \cong H$  and $H \cong I$, then $G \cong I$.\\
If $G \cong H$ and $H$ is \textit{abelian}, then $G$ is albelian.\\
If $G \cong H$ and $H$ is \textit{cyclic}, then $H$ is cyclic.\\

\subtitle{Proposition 9.04 - }{Indentity element and Isomorphisms}
Let $\phi : G \to H$ be an isomorphism, $e_G$ \& $e_H$ be the identity elements of these groups and $x \in G$. Then
\begin{enumerate}[label=\roman*)]
  \item $\phi(e_G) = e_H$;
  \item $\phi(x^{-1}) = \phi(x)^{-1}$;
  \item $\phi(x^i) = \phi(x)^i,\quad \forall\ i \in \integers$; and,
  \item $ord_G(x) = ord_H(\phi(x))$.\\
\end{enumerate}

\subtitle{Proposition 9.05 - }{Order of Isomorphic Groups}
Let $G$ \& $H$ be isomorphic then $|G| = |H|$.\\

\subtitle{Proposition 9.06 - }{Order of Elements of Isomorphic Groups}
Let $G$ \& $H$ be isomorphic and $n \in \nat$.\\
Then $G$ and $H$ have the same number of elements of order $n$.

\section{Direct Product}

\subtitle{Definition 10.01 - }{Direct Product}
Let $G$ \& $H$ be groups with the same binary operator.\\
The \textit{direct product}, $G \times H$, is the cartesian product of the sets of $G$ and $H$ with the binary operator
$$(x, y)(x', y') = (xx', yy'),\quad x, x' \in G\ y, y' \in H$$

\subtitle{Proposition 10.02 - }{Direct Product as a group}
The direct product of two groups is itself a group.\\

\subtitle{Proposition 10.03 - }{Properties of Direct Product}
Let $G$ and $H$ be groups with the same binary operator.
\begin{enumerate}[label=\roman*)]
  \item $G \times H$ is \textit{infinite} iff both $G$ and $H$ are infinite;
  \item $G \times H$ is \textit{abelian} iff both $G$ and $H$ are abelian; and,
  \item If $G \times H$ is \textit{cyclic}, \underline{then} $G$ and $H$ are cyclic.\\
\end{enumerate}

\subtitle{Proposition 10.04 - }{Order of Elements of Direct Product}
Let $g \in G, h \in H$ with $ord_G(g) = m \in \nat$ and $ord_H(h) = n \in \nat$ then for $(g, h) \in G \times H$
$$ord_{G \times H}(g, h) = lcm(m, n)$$

\subtitle{Theorem 10.05 - }{Cycle Direct Products}
Let $G\ \&\ H$ be finite cyclic groups.\\
Then $G \times H$ is a cyclic group iff $gcd(|G|, |H|) = 1$.\\

\subtitle{Definition 10.06 - }{Klein 4-Group}
A \textit{Klein 4-Group} is a group of order 4 such that every element, except the identity, has order 2.\\

\subtitle{Proposition 10.07 - }{}
Let $m, n \in \nat$ such that $gcd(m,n) = 1$. Then
$$U_{mn} \cong U_m \times U_n$$

\section{Lagrange's Theorem}

\subtitle{Theorem 11.01 - }{Lagrange's Theorem}
Let $G$ be a finite group, and $H \leq G$, then $|H|$ divides $|G|$.\\

\subtitle{Definition 11.02 - }{Co-Sets}
Let $G$ be a group, $H \leq G$ and $x \in G$.\\
The \textit{left co-set} is defined as $xH = \{xh \in G : h \in H\} \subseteq G$.\\
The \textit{right co-set} is defined as $Hx = \{hx \in G : h \in H\} \subseteq G$.\\

\subtitle{Theorem 11.03 - }{Order of Co-set}
There exists a bijection, $\phi : H \to xH$, where $\phi(h) = xh$ so $$|H| = |xH|$$

\subtitle{Theorem 11.04 - }{Relationship between Co-sets}
Let $x, y \in G$ and $H \leq G$ then either
$$xH = yH \mathrm{\ or\ } xH \cap yh = \emptyset$$

\subtitle{Theorem 11.05 - }{Cosets of Abelian Groups}
Let $G$ be an abelian group and $H \leq G$ then $xH = Hx$.\\

\subtitle{Definition 11.06 - }{Index}
Let $H \leq G$.\\
Then \textit{index}, $|G : H|$, is the number of left co-sets, $xH$, in $G$.

\section{Some Consequences and Applications of\\ Lagrange's Theorem}

\subtitle{Propostion 12.01 - }{Lagrange for Order of Elements}
Let $G$ be a finite group with $|G| = n$.\\
Then $\forall\ x \in G,\ ord(x) | n$ meaning $x^n = e$.\\

\subtitle{Theorem 12.02 - }{Fermat's Little Theorem}
Let $p \in \nat$ and $a \in \integers$ such that $p \not | a$. Then
$$a^{p-1} \equiv 1 (mod\ p)$$

\subtitle{Definition 12.03 - }{Euler's Phi Function}
\textit{Euler's phi function} is the function, $\phi : \nat \to \nat$, where
$$\phi(m) = |\{ a \in \integers : 0 \leq a \leq m; gcd(a, m) = 1 \}|$$

\subtitle{Theorem 12.04 - }{Fermat-Euler Theorem}
Let $m > 0$ and $a \in \integers$ with $gcd(a, m) = 1$. Then
$$a^{\phi(m)} \equiv 1 (mod\ m)$$

\subtitle{Theorem 12.05 - }{Properties of Prime-Ordered Groups}
Let $p \in \nat$ be prime and $G$ be a group such that $|G| = p$. Then
\begin{enumerate}[label=\roman*)]
  \item $G$ is cyclic;
  \item $\forall\ x \in G \backslash \{e\}, ord(x) = p$ and $G = \langle x \rangle$; and,
  \item $G$ only has two subgroups, both trivial, $\{e\}$ and $G$ itself.\\
\end{enumerate}

\subtitle{Proposition 12.06 - }{Relationship between Prime-Ordered Subgroups}
Let $p \in \nat$ be prime and $H, I \leq G$ such that $|H| = p = |I|$, then either
$$P = Q \mathrm{\ or \ } P \cap Q = \{e\} $$

\subtitle{Proposition 12.07 - }{Relationship between Relatively-Prime-Ordered Subgroups}
Let $m, n \in \nat$, with $gcd(m,n) = 1$ and $H, I \leq G$ such that $|H| = m, |I| = n$, then
$$H \cap I = \{e\}$$

\subtitle{Theorem 12.08 - }{Odd Primed-Ordered Groups}
Let $p \in \nat$ be an odd-prime. Then
\begin{enumerate}[label=\roman*)]
  \item Every group of order $2p$ is either \textit{cyclic} or \textit{isomorphic} to $D_{2p}$; and,
  \item Every group of order $p^2$ is either \textit{cyclic} or \textit{isomorphic} to $(\frac{\integers}{p\integers}) \times (\frac{\integers}{p\integers})$.
\end{enumerate}

\section{Symmetric Groups}

\subtitle{Definition 13.01 - }{Symmetric Group}
Let $X$ be a set.\\
The \textit{symmetric group} on $X$ is the group, $S(X)$, of all permutations of $X$ under composition.\\
\underline{N.B.} - $S_n$ is the group of all permutations of $\{1 , \dots , n\}$.\\

\textbf{Proposition 13.02 - }\textit{Order of Symmetric Group}
$$|S_n| = n!$$

\subtitle{Definition 13.03 - }{k-cycle}
A \textit{k-cycle} in $S_n$, where $k \leq n$, is a permutation where
$$\sigma(x_i) = x_{i+1},\quad \sigma(x_k) = x_1$$
\underline{N.B.} - denoted by $\sigma = (x_1\ x_2\ \dots\ x_k)$.\\

\subtitle{Theorem 13.04 - }{Order of a k-cycle}
Let $\sigma$ be a k-cycle then $ord_{S_n}(\sigma) = k$.\\

\subtitle{Definition 13.05 - }{Transposition}
A \textit{transposition} is a permutation that swaps two-elements and leaves all other elements unchanged.\\
\underline{N.B.} - $\sigma(x_m) = x_n,\ \sigma(x_n) = x_m$ is denoted by $\sigma = (x_m, x_n)$. This can be extended for any number of elements.\\

\subtitle{Definition 13.06 - }{Disjoint Cycles}
Disjoint cycles are cycles of $S_n$ that have no elements share a common position.\\

\subtitle{Theorem 13.07 - }{Order of Products of Disjoint Cycles}
Let $f$ be the product of disjoint cycles of length $k_1, k_2, \dots , k_n$ then
$$ord(f) = lcm(k_1, k_2, \dots , k_n)$$

\end{document}
