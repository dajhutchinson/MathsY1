\documentclass[11pt]{article}

\usepackage[margin=1in, paperwidth=8.3in, paperheight=11.7in]{geometry}
\usepackage{amsfonts}
\usepackage{amsmath}

\begin{document}

\title{Introduction to Group Theory - Problem Sheet 4}
\author{Dom Hutchinson}
\maketitle

\begin{flushleft} 1) a) Find the orders of all elements of $\displaystyle{D_{12}}$, the dihedral group of order 12.
\end{flushleft}

\begin{center}
$D_{12}=\left\{e,a,a^2,a^3,a^4,a^5,b,ab,a^2b,a^3b,a^4b,a^5b\right\}$
\end{center}
\begin{eqnarray*}
(a^ib)^2=(a^ib)(a^ib)=a^iba^ib&=&a^ia^{-i}bb=ee=e\ \forall\ i \in \mathbb{N}\\
\\ord(e)=1,&&ord(b)=2,\\
ord(a)=6,&&ord(ab)=2,\\
ord(a^2)=3,&&ord(a^2b)=2,\\
ord(a^3)=2,&&ord(a^3b)=2\\
ord(a^4)=3,&&ord(a^4b)=2\\
ord(a^5)=6,&&ord(a^5b)=2
\end{eqnarray*}

\begin{flushleft}\ \ \  b) Let $G$ be the group of symmetries of a cube. Describe, geometrically, an element of $G$ of order 2, and element of order 3, and an element of order 4.\end{flushleft}

Rotating $\pi\ rads$ about the x-axis has order 2;\\
\indent Rotating $\frac{\pi}{2}\ rads$ about x-axis, then $\frac{\pi}{2}\ rads$ about the y-axis has order 3;\\
\indent Rotating $\frac{\pi}{2}\ rads$ about x-axis has order 4.

\newpage
\begin{flushleft} 2) a) Let $G$ be an \textbf{abelian} (multiplicatively written) group with elements $x,y$ with\\
\begin{center}
ord($x$)$=m<\infty$ and ord($y$)=$n<\infty$
\end{center}
Show that if $m$ and $n$ both divide $k$, then $(xy)^k=2$, and deduce that ord($xy$) divides lcm($m,n$).
\end{flushleft}

\begin{eqnarray*}
&&\mathrm{As}\ m|k\ \mathrm{and}\ n|k\\
&&\mathrm{Then}\ \exists\ a,b \in \mathbb{N}\ \mathrm{st}\ am=k=bn\\
\\&&\mathrm{Since\ ord(}x\mathrm{)}=m\mathrm{\ and\ ord(}y\mathrm{)}=n\\
&&\mathrm{Then}\ x^m=e=y^n\\
\\(xy)^k&=&x^ky^k\\
&=&a^{am}b^{bn}\ \mathrm{as\ \textit{G}\ is\ abelian}\\
&=&(x^m)^a(y^n)^b\\
&=&e^ae^b\\
&=&\underline{e}
\end{eqnarray*}

{\huge WIP}

\begin{flushleft}\ \ \  b) By finding an example, show taht what you were asked to prove in a) is not true for \textbf{non-abelian} groups.
\end{flushleft}
\begin{eqnarray*}
(2 1 3), (2 1 3)\ \in\ S_3&&\ \mathrm{which\ is\ not\ abelian.}\\
\mathrm{ord}((2 1 3))&=&2\\
\mathrm{ord}((1 3 2))&=&2\\
\mathrm{lcm}((2,2))&=&2\\
\\(2 1 3)(1 3 2)&=&(2 3 1)\\
\mathrm{ord}((2 3 1))&=&3\\
=>\mathrm{ord}((1 3 2))&>&\mathrm{lcm}(2,2)\\
=>\mathrm{ord}((1 3 2))&\not|&\mathrm{lcm}(2,2)\\
\end{eqnarray*}

\begin{flushleft} 3) Let $G$ be a, multiplicatively written, group and $g,x \in G$. Prove that ord($x$)\ =\ ord($gxg^-1$)
\end{flushleft}

\begin{center}
Let $n \in \mathbb{N}$ st it is the lowest value where $(gxg^{-1})^n=e$
\end{center}
\begin{eqnarray*}
&=>&(gxg^{-1})(gxg^{-1})...(gxg^{-1})=e\\
&=>&gx(g^{-1}g)x(g^{-1}g)...(g^{-1}g)xg^{-1}=e\\
&=>&gx^ng^{-1}=e\\
&=>&g^{-1}gx^ng^{-1}g=g^{-1}eg\\
&=>&ex^ne=g^{-1}g\\
&=>&x^n=e
\end{eqnarray*}

\newpage
\begin{flushleft} 4) Find all the cyclic subgroups of $D_{12}$
\end{flushleft}
\begin{eqnarray*}
&&D_{12}=\left\{e,a,a^2,a^3,a^4,a^5,b,ab,a^2b,a^3b,a^4b,a^5b\right\}\\
&&\langle e \rangle ,\langle a \rangle ,\langle a^2 \rangle ,\langle a^3 \rangle ,\langle a^4 \rangle ,\langle a^5 \rangle ,\langle b \rangle ,\langle ab \rangle ,\langle a^2b \rangle ,\langle a^3b \rangle ,\langle a^4b \rangle ,\langle a^5b \rangle \le D_{12}
\end{eqnarray*}

\begin{flushleft}
 5) Let $G$ be the group of permutations of $\{ 1,2,3,4,5 \}$. Find an element of $G$ with order 6.
\end{flushleft}
\begin{eqnarray*}
\mathrm{Let}\ I &=&
\begin{pmatrix}
1 & 2 & 3 & 4 & 5\\
2 & 3 & 1 & 5 & 4
\end{pmatrix}\\
e&=&(12345)\\
I&=&(31254)\\
I^2&=&(23145)\\
I^3&=&(12354)\\
I^4&=&(31245)\\
I^5&=&(23154)\\
I^6&=&(12345)=e\\
&=>&\underline{\mathrm{ord}(I)=6}
\end{eqnarray*}

\begin{flushleft}
 6) Show that the group $(\mathbb{Q}, +)$ of rational numbers under addition is not cyclic.
\end{flushleft}
\begin{eqnarray*}
&&\text{Suppose } x \in \mathbb{Q} \text{ st } \langle x \rangle = G\\
%
&&\text{As } x \in \mathbb{Q} \text{ then } \exists \text{ st } a \in \mathbb{Z}, b \in \mathbb{N} \text{ st } x=\frac{a}{b}\\
%
&&\text{Thus } \frac{x}{2} = \frac{a}{2b} \in \mathbb{Q}\\
%
&&\langle x \rangle = \langle \frac{a}{b} \rangle\\
%
&&= \left\{...,\frac{a}{b}-2.\frac{a}{b}, \frac{a}{b}-\frac{a}{b},\frac{a}{b},\frac{a}{b}+\frac{a}{b},...\right\}\\
%
&&=\left\{...,-\frac{a}{b},0,\frac{a}{b},\frac{2a}{b},...\right\}\\
%
&&\text{As } \frac{a}{2b} \not \in \langle \frac{a}{b} \rangle\\
%
&&\text{Then } \frac{x}{2} \not \in \langle x \rangle\\
%
&&\text{So } \langle x \rangle\ \text{does not contain all rationals thus }\langle x \rangle \not = G\ \forall\ x \in \mathbb{Q}
%
\end{eqnarray*}
\end{document}